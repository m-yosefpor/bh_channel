%%%%%%%%%%%%%%%%%%%%%%%%%%%%

\section*{چکیده}
\begin{spacing}{2}

پارادوکس اطلاعات سیاه چاله ها که به بررسی سرنوشت اطلاعات سیاه چاله ها می پردازد، در اواسط قرن بیستم توسط هاوکینگ مطرح شده بود و مدت بسیار زیادی ذهن دانشمندان را به خود مشغول کرده بود. همان طور که اکثر تئوری های مهم از  توجیه پارادوکس ها شکل گرفته است، خیلی از دانشمندان فکر می کردند که برای توجیه این پارادوکس نیاز است تا صحت تئوری های فیزیکی موجود دوباره بررسی شود.

تلاش های زیادی با روش های متفاوت برای حل این مساله پیشنهاد شده است. با پیشرفت تئوری اطلاعات کوانتمی در ابتدای قرن بیست و یکم، بار دیگر این مساله از دیدگاه  تئوری اطلاعات کوانتمی به طور دقیق مورد بررسی قرار گرفت تا بتوان با مدل کردن سیاه چاله ها به عنوان کانال های کوانتمی مخابراتی  و محاسبه ی ظرفیت این کانال ها، به این پارادوکس پاسخ داد.

در این پروژه، ابتدا پارادوکس اطلاعات سیاه چاله ها تشریح شده است. سپس ابزار های تئوری اطلاعات مورد نیاز برای بررسی این موضوع معرفی شده و با استفاده از آن ها تحول سیاه چاله ها با کانال مخابراتی مدل شده است. سپس ظرفیت کلاسیکی و ظرفیت کوانتمی آن محاسبه شده و طبق آن به توجیه پارادوکس پرداخته شده است.
\\
\textbf{واژه‌های کلیدی:}
تئوری اطلاعات کلاسیک، تئوری اطلاعات کوانتمی، پارادوکس اطلاعات سیاه چاله ها، ظرفیت کوانتمی کانال، ظرفیت کلاسیکی کانال، وفاداری، تابش هاوکینگ. 
\end{spacing}

\newpage\null\newpage
