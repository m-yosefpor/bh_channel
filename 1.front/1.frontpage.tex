%%%%%%%%%%%%%%%%%%%%%%%%%%%%%%%%%%%%%%%%
\newgeometry{margin=2.5cm}
\pagestyle{empty}
%%%%%%%%%%%%%%%%%%%%%%%%%%%%%%%%%%%%%%%%
\begin{center}
\includegraphics[width=0.3\textwidth]{logo.png}\\
\Nastaliq
\large{دانشکده مهندسی برق}\\\vspace{1cm}
\titlefont
\large{پایان‌نامه دوره کارشناسی}\\
\large{در رشته مهندسی برق- گرایش مخابرات}\\
\vspace{1cm}
\Large{کاربرد تئوری اطلاعات کلاسیک و کوانتمی در بررسی اطلاعات سیاه چاله به منظور محاسبه و شبیه سازی ظرفیت کوانتمی کانال سیاه چاله}\\\vspace{2cm}
\normalsize{نام دانشجو:}\\
\large{محمد یوسف پور}\\\vspace{2cm}
\normalsize{استاد راهنما:}\\
\large{دکتر بهاره اخباری}\\\vspace{2cm}
\normalsize{پاییز 1396}

\end{center}
%\newpage\null\thispagestyle{empty}\newpage
\newpage\null\newpage
