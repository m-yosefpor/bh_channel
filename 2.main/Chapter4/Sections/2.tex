\section{پیشنهادات}
%%%%%%%%%%%%%%%%%%%%%%%%%%%%%%%%%%%%%%%%%%%%%%%%%%%%%%%%%%%%%%
\subsection{ادامه ی کار در بررسی پارادوکس}
یکی از پیشنهاد های موجود برای ادامه ی کار، این است که ظرفیت سیاه چاله در حالت هایی که در مود های دیر هنگام بررسی نشد، بررسی شود. دیدیم که ظرفیت کلاسیکی سیاه چاله در مود های دیر هنگام در حالت حدی
$g'_k = g_k$
که سیاه چاله کاملا بازتاب کننده است،
بررسی شد و گفتیم که انتظار داریم که نتیجه برای حالت کلی نیز برقرار باشد. همینطور در محاسبه ی ظرفیت کوانتمی دیدیم که دو حالت حدی کاملا بازتاب کننده و کاملا جذب کننده بررسی شد. اگر بخواهیم در حالت های کلی تر این مسئله بررسی شود، باید ظرفیت کانال را در غیر از این حالات نیز محاسبه کرد. مشکلی که  در این روش با آن روبرو هستیم این است که ظرفیت کانال های کوانتمی چند حرفی است و در حالت کلی هنوز نمی توانیم آن را محاسبه کنیم. زمانی حتی ظرفیت کانال هایی که امروزه ظرفیت آن ها را میدانیم (مثل کانال های نازل شونده)
نیز قابل محاسبه نبود و با پیشرفت تئوری اطلاعات کوانتمی، محاسبه ی این ظرفیت ها ممکن شد. برای اینکه بتوانیم این مسئله را حل کنیم باید ظرفیت کانال های کوانتمی را برای دسته های بیشتری پیدا کنیم. همچنین، حتی اگر قادر نیستیم ظرفیت را محاسبه کنیم، شاید بتوان برای حالت های کلی، کران بالا و پایین ظرفیت را محاسبه کنیم.
%%%%%%%%%%%%%%%%%%%%%%%%%%%%%%%%%%%%%%%%%%%%%%%%%%%%%%%%%%%%%%
\subsection{وجود شخص سوم شنود گر و مخابرات چند کاربره}
با توجه به اینکه از مفاهیم ظرفیت، کدینگ و کد های تصحیح خطا در پارادوکس سیاه چاله ها استفاده می شود، شاید بتوان بررسی کرد که مفاهیم شخص شنود گر و حتی مخابرات چند کاربره را وارد این مساله کنیم تا بتوان به نتایجی بهتر دست پیدا کرد. البته باید دید که این مفاهیم در دنیای فیزیکی معادل چه پدیده هایی هستند و این به بررسی بیشتر در نحوه ی کار سیاه چاله ها بستگی دارد.
%%%%%%%%%%%%%%%%%%%%%%%%%%%%%%%%%%%%%%%%%%%%%%%%%%%%%%%%%%%%%%
\subsection{کاربرد های کوانتم اپتیک}
یکی دیگر از کاربرد های  این ابزار ها و ظرفیت هایی که در اینجا محاسبه کردیم، در کوانتم اپتیک است. با توجه به اینکه همیلتونی در آن سیستم ها مشابه این مساله است، لذا نتایجی  را که برای ظرفیت و وفاداری گرفتیم، ممکن است بتوانیم در برخی کاربرد های کوانتم اپتیکی تعمیم دهیم.
%%%%%%%%%%%%%%%%%%%%%%%%%%%%%%%%%%%%%%%%%%%%%%%%%%%%%%%%%%%%%%
\subsection{ظرفیت کانال ها در امنیت کوانتمی و توافق کلید}
اگر چه کانال های ذکر شده در اینجا به خاطر مدل سیاه چاله مطرح شد،
مثل کانال های
\lr{Unruh}
و
\lr{depolarizing}
و ...
 اما این ها کانال های کلی مخابراتی هستند که ممکن است در سیستم های مخابرات کوانتمی رخ بدهد. یکی از کاربرد های تئوری اطلاعات کوانتمی در امنیت کوانتمی و توافق کلید می باشد. در این روش، می توان کلیدی را توسط پروتکل های مختلفی مانند
 \lr{BB84}
 و ... انتقال دهیم و سپس با استفاده از آن، رمز نگاری 
 \lr{OTP}
 \LTRfootnote{One-Time Padding}
 انجام دهیم تا اطلاعات به صورت کاملا امن انتقال پیدا کند. برای اینکه نرخ انتقال اطلاعات را در مخابرات کوانتمی به دست آوریم، باید ظرفیت کانال را محاسبه کنیم. یکی دیگر از راه های ادامه ی کار، بررسی کانال های موجود در مخابرات کوانتمی و محاسبه ی ظرفیت آن ها با استفاده از تکنیک های مطرح شده در این پایان نامه می باشد.
 %%%%%%%%%%%%%%%%%%%%%%%%%%%%%%%%%%%%%%%%%%%%%%%%%%%%%%%%%%%%%%