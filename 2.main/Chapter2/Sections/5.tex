\section{قضیه های اطلاعات کوانتمی}
%%%%%%%%%%%%%%%%%%%%%%%%%%%%%%%%%%%%%%%%%%%%%%%%%%%%%%%%%%%%%%%%%%%%%%%%%%%
در این قسمت به قضیه های مهم مورد استفاده تئوری اطلاعات کوانتمی می پردازیم. این قضایا به قضایای 
\lr{no-go}
معروف اند. بدین معنی که این قضایا بیان می کنند که وضعیت خاصی از نظر فیزیک به هیچ وجه ممکن نمی باشند. 
%%%%%%%%%%%%%%%%%%%%%%%%%%%%%%%%%%%%%%%%%%%%%%%%%%%%%%%%%%%%%%%%%%%%%%%%%%%
\subsection{قضیه ی \lr{no-cloning}}

هیچ عملگر یکانی
$U$
روی
$\hi \otimes \hi$
وجود ندارد به طوری که برای همه ی حالت های به هنجار 
$| \phi \rangle_A$
و
$| e \rangle_B$
در
$\hi$

\begin{equation}
	U(| \phi \rangle_A | e \rangle_B) = e^{i\alpha (\phi,e)} | \phi \rangle_A | e \rangle_B
\end{equation}

این قضیه بیان می کند که غیر ممکن است که از یک حالت کوانتمی دلخواه نامشخص کپی دقیقا مشابهی تولید کنیم. معمولا قضیه ی 
\lr{no-cloning}
برای حالت های خالص بیان و اثبات می شود.




%%%%%%%%%%%%%%%%%%%%%%%%%%%%%%%%%%%%%%%%%%%%%%%%%%%%%%%%%%%%%%%%%%%%%%%%%%%
\subsection{قضیه ی \lr{no-broadcast}}
 قضیه ی 
 \lr{no-broadcast}
 به نوعی تعمیم قضیه ی
 \lr{no-cloning}
  برای حالت های مخلوط را بیان می کند.
  
  اگر حالت اولیه ی
  $\rho_1$
  را داشته باشیم، غیر ممکن است که بتوانیم حالت
  $\rho^{AB}$
  را در فضای هیلبرت
  $\hi_A \otimes \hi_B$
  ایجاد کنیم به طوری که 
  \begin{equation}
  	tr_A \rho^{AB} = \rho_1 \quad,\quad tr_B \rho^{AB}=\rho_1
  \end{equation}
  البته اگر تعداد بیشتری از یک کپی در ورودی داشته باشیم، قضیه برقرار نمی باشد. به عنوان مثال می توان شش کپی از چهار کپی ورودی ایجاد کرد.
%%%%%%%%%%%%%%%%%%%%%%%%%%%%%%%%%%%%%%%%%%%%%%%%%%%%%%%%%%%%%%%%%%%%%%%%%%%
\subsection{قضیه ی \lr{no-deleting}}

فرض کنید که 
$\pk$
یک حالت در فضای هیلبرت
$\hi$
باشد. در این صورت هیچ تبدیل یکانی 
$U$
وجود ندارد که:
\begin{equation}
	U(\pk \pk |\phi\rangle) = \pk |0\rangle |\phi '\rangle
\end{equation}


در واقع این قضیه بیان می کند که اگر دو کپی از یک حالت ناشناخته داشته باشیم، نمی توانیم یکی از این حالات را حذف کنیم. این قضیه یک دوگان زمانی برای قضیه ی 
\lr{no-cloning}
می باشد. که هر دو اینها وابسته به خطی بودن مکانیک کوانتمی می باشند.

%%%%%%%%%%%%%%%%%%%%%%%%%%%%%%%%%%%%%%%%%%%%%%%%%%%%%%%%%%%%%%%%%%%%%%%%%%%
\subsection{قضیه ی \lr{no-communication}}

فرض کنید که سیستم ترکیبی بین آلیس 
و
باب
با عملگر چگالی 
$\rho^{AB} \in \hi_A \otimes \hi_B$
توصیف شود. آلیس یک اندازه گیری محلی روی زیر سیستم خودش انجام می دهد. به طور کلی این با یک عملگر کوانتمی به صورت زیر توصیف می شود:
\begin{equation}
	P(\sigma) = \sum_k(V_k \otimes I_{\hi_B})^\dagger \sigma (V_k \otimes I_{\hi_B}) 
\end{equation}

که 
$V_k$
ماتریس های 
$Kraus$
هستند به طوریکه
$\sum_k V_k V_k^\dagger = I_{\hi_A}$

پس از این اندازه گیری، سیستم باب به صورت 
$tr_{\hi_A} (P(\sigma))$
 در می آید. می توان نشان داد
 \cite{wilde}
 : 
 \begin{equation}
 	tr_{\hi_A} (P(\sigma)) = tr_{\hi_A} (\sigma)
 \end{equation}
 در نتیجه باب نمی تواند تفاوت بین آنچه آلیس انجام داده است یا یک اندازه گیری تصادفی یا حتی اگر اصلا هیچ عملیاتی انجام نشده باشد، را متوجه شود.

این قضیه بیان می کند که ناممکن است که  یک مشاهده گر  با اندازه گیری روی یک زیر سیستم از یک سیستم کلی در هم تنیده، بتواند به مشاهده گر دیگر اطلاعاتی مخابره کند. این قضیه از به وجود آمدن، پارادوکس
\lr{EPR}
\LTRfootnote{Einstein–Podolsky–Rosen paradox} 
پیش گیری می کند.


%%%%%%%%%%%%%%%%%%%%%%%%%%%%%%%%%%%%%%%%%%%%%%%%%%%%%%%%%%%%%%%%%%%%%%%%%%%
\subsection{قضیه ی \lr{no-teleportation}}

این قضیه بیان می کند که به هیچ وجه نمی توان یک حالت کوانتمی را به صورت رشته ای از بیت های کلاسیکی بیان کرد و برعکس. به عبارتی نمی توان یک حالت کوانتمی را به یک رشته بیت کلاسیک تبدیل کرد و با انتقال آن ها بتوان حالت کوانتمی را بازیابی کرد. به این خاطر به این قضیه 
\lr{no-teleportation}
می گویند. توجه شود که نباید این مطلب را با تلپورت کوانتمی
\LTRfootnote{Quantum Teleportation}
اشتباه گرفت که بیان می کند که می توان یک حالت را در یک مکان نابود کرد و یک کپی مشابه آن در یک مکان دیگر ایجاد کرد.

در مقابل، بر عکس این کار، یعنی تبدیل اطلاعات کلاسیک به اطلاعات کوانتمی و انتقال آن و سپس بازیابی کامل اطلاعات کلاسیک امکان پذیر است. این کار با کد کردن آن در پایه های عمود که همیشه تمیز پذیر هستند، امکان پذیر است.
%%%%%%%%%%%%%%%%%%%%%%%%%%%%%%%%%%%%%%%%%%%%%%%%%%%%%%%%%%%%%%%%%%%%%%%%%%%


%%%%%%%%%%%%%%%%%%%%%%%%%%%%%%%%%%%%%%%%%%%%%%%%%%%%%%%%%%%%%%%%%%%%%%%%%%%
\subsection{قضیه ی \lr{no-hiding}}

یکی از مهمترین قضیه های تئوری اطلاعات کوانتمی که در بحث پارادوکس اطلاعات سیاه چاله ها استفاده می شود، قضیه ی 
\lr{no-hiding}
می باشد.

وقتی سیستمی با یک محیط خارجی در تعامل باشد، نا منسجم 
\LTRfootnote{Decoherent}
می شود و در مواردی ممکن است سیستم اصلی اطلاعاتش را به کلی از دست بدهد. حال باید دید که سرنوشت این اطلاعات چه شده است و آیا این اطلاعات در محیط رفته است یا در همبستگی بین محیط و سیستم؟ قضیه ی  
\lr{no-hiding}
اثبات می کند که اطلاعات از دست رفته از سیستم به طور کامل به زیر سیستمی از محیط رفته است و در همبستگی محیط و سیستم نیست.

حالت خالص 
$\pk$
را در فضای هیلبرت
$\hi$
در نظر بگیرید. فرض کنید  طی یک فرایند فیزیکی حالت
$\pk \pb$
به حالت مخلوط
$\rho = \sum_k p_k |k\rangle \langle k|$
تبدیل شود. اگر حالت اولیه محیط به صورت 
$|A\rangle$
باشد. آن گاه در فضای هیلبرت کل، این نگاشت به صورت زیر خواهد بود:
\begin{equation}
	\pk \otimes |A\rangle \rightarrow \sum_k \sqrt{p_k} |k\rangle \otimes |A_k(\psi)\rangle = \sum_k \sqrt{p_k} |k\rangle \otimes (|q_k\rangle \otimes \pk \oplus 0)
\end{equation}

که در آن 
$|A_k(\psi)\rangle$
پایه ی راست بهنجار فضای هیلبرت محیط می باشد
\cite{nohidden}.


این مساله به شدت در پارادوکس اطلاعات سیاه چاله ها و به طور کلی در هر فرایندی که  ممکن است اطلاعات از دست بدهد مطرح است و بیان می کند در هیچ تحول فیزیکی، ناممکن است که اطلاعات سیستم ایزوله از دست برود.  لذا  این به نوعی یک پایستگی اطلاعات کوانتمی در سیستم ایزوله را بیان می کند، بدین معنی که اطلاعات کوانتمی نه به وجود می آید و نه از دست می رود. اگر چه این نتیجه در دو قضیه ی 
\lr{no-cloning}
و
\lr{no-deleting}
هم ریشه دارد، اما قضیه ی 
\lr{no-hiding}
اثبات نهایی پایستگی اطلاعات کوانتمی می باشد. اهمیت این قضیه به خاطر این است که پایستگی تابع موج در مکانیک کوانتمی را اثبات می کند. به عنوان مثال اگر ضرایب احتمال در یک سیستم نابود شود، در سیستم دیگری ظاهر خواهد شد. با توجه به اینکه تابع موج کلیه ی اطلاعات موجود در یک سیستم را شامل می شود، لذا پایستگی آن معادل پاستگی اطلاعات کوانتمی می باشد.