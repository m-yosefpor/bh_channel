\section{مکانیک کوانتمی}
%%%%%%%%%%%%%%%%%%%%%%%%%%%%%%%%%%%%%%%%%%%%%%%%%%%%%%%%%%%%%%%%%%%%%%%%%%%
\subsection{دیدگاه کوانتمی}
در مکانیک کلاسیک، مفهوم اندازه گیری معنی خاصی نداشت، به این خاطر که متغیر های 
$x$
و 
$p$
که حالت سیستم در فضای فاز را مشخص می کنند، در هر لحظه ای مشخص است. به عبارت دیگر، مشاهده گر همیشه می تواند اطلاعات و حالت دقیق سیستم را در هر لحظه داشته باشد و دنبال کند. اما در مکانیک کلاسیک، قبل از مشاهده کردن، مقدار هر مشاهده پذیر مشخص نیست. لذا حالت سیستم را به عنوان برداری در فضای هیلبرت توصیف می کنیم که تصویر این بردار روی هر یک از بردار های ویژه ی عملگر های مشاهده پذیر های مان، با  این احتمال متناسب است که نتیجه اندازه گیری آن مشاهده پذیر مقدار ویژه ی متناظر با آن بردار ویژه باشد. لذا، نتایج هر اندازه گیری فقط پس از اندازه گیری مشخص می شود و تا قبل از آن فقط می توان در مورد احتمال آن بحث کرد.

بر خلاف مکانیک کلاسیک که یک نظریه ی 
\lr{objective}
می باشد و حالت سیستم به مشاهده پذیر بستگی ندارد، در مکانیک کوانتمی، اطلاعات مشاهده گر می تواند حالت سیستم را عوض کند و به نوعی یک نظریه ی
\lr{subjective}
می باشد که ناظر های مختلف ممکن است حالت های متفاوتی را برای یک سیستم معرفی کنند.
%%%%%%%%%%%%%%%%%%%%%%%%%%%%%%%%%%%%%%%%%%%%%%%%%%%%%%%%%%%%%%%%%%%%%%%%%%%
\subsection{حالت کوانتمی}
همان طور که گفته شد، هر حالت کوانتمی به عنوان یک بردار در فضای هیلبرت در نظر گرفته می شود. دیراک هر بردار را به صورت یک کِت با نماد
$| \psi \rangle $
در نظر گرفت. اگر برای این فضای هیلبرت پایه ای در نظر گرفته شود، می توان هر حالت را به صورت یک ماتریس تک ستونه نشان بدهیم.
\begin{equation*}
	\begin{pmatrix}
		\psi_1 \\ \psi_2 \\ \vdots \\ \psi_n
	\end{pmatrix}
\end{equation*}
%%%%%%%%%%%%%%%%%%%%%%%%%%%%%%%%%%%%%%%%%%%%%%%%%%%%%%%%%%%%%%%%%%%%%%%%%%%
\subsection{اندازه گیری}
هر مشاهده پذیر در مکانیک کوانتمی با یک عملگر  یکانی در فضای هیلبرت مشخص می شود که مقادیر ممکن اندازه گیری مقدار های  ویژه آن عملگر اند و  بردار های  ویژه آن عملگر، حالاتی هستند که اگر سیستم در آن باشد، نتیجه اندازه گیری آن مشاهده پذیر، قطعا  مقدار  ویژه متناظر با آن می باشد. با توجه به این که مقادیر ویژه ی عملگر های یکانی، حقیقی اند، لذا آزمایش یک مقدار حقیقی خواهد شد. علاوه بر این، بردار های ویژه ی آن عملگر برای فضای هیلبرت تشکیل یک پایه می دهند (نه لزوما یکتا، چون ممکن است تبهگن باشد). اگر حالت مورد نظر را در پایه ی تشکیل شده توسط بردار های  ویژه این عملگر بنویسیم، ضرایب این بسط متناسب با احتمال مشاهده  مقدار ویژه متناظر با هر یک از بردار های پایه می باشد.
%%%%%%%%%%%%%%%%%%%%%%%%%%%%%%%%%%%%%%%%%%%%%%%%%%%%%%%%%%%%%%%%%%%%%%%%%%%
\subsection{اصول موضوعه مکانیک کوانتمی}
با در نظر گرفتن توضیحات بالا، می توان به اصول موضوعه ی مکانیک کوانتمی پرداخت. البته این اصول موضوعه در منبع های متفاوت شکل های متفاوتی دارند، که البته به یکدیگر کاهش پیدا می کنند. ما در اینجا مناسب ترین نوع اصول موضوعه متناسب با کاربرد تئوری اطلاعاتی را در نظر گرفته ایم
\cite{ipm, shankar}
.
\subsubsection{اصل اول: فضای حالت}
به هر سیستم فیزیکی یک فضای هیلبرت متناظر است. حالت سیستم (در هر لحظه از زمان) با یک بردار با طول واحد در فضاي هیلبرت مشخص می شود. دو بردار که ضریبی از یکدیگر باشند یک حالت فیزیکی را بیان می کنند.

\subsubsection{اصل دوم: اندازه گیری}
اگر سیستم در حالت
$\pk$
باشد، اندازه گیری مشاهده پذیری روی آن باعث می شود که نتیجه یکی از مقادیر ویژه
$w$
به احتمال 
$p(w) \propto | \langle w \pk |^2$
خواهد بود و حالت سیستم به  بردار ویژه
$| w \rangle$
فرو می پاشد
\LTRfootnote{Collapse}
.

\subsubsection{اصل سوم: جابجا گری}
جابجاگری عمگر های تکانه
$\hat{P}$
 و عملگر مکان
 $\hat{X}$
  صفر نمی باشد، بلکه :
\input{\formulaPATH{2}{10}}
که 
$\hbar$
ثابت پلانک است.
%%%%%%%%%%%%%%%%
\subsubsection{اصل چهارم: تحول زمانی}
تحول زمانی یک سیستم ایزوله با یک عملگر یکانی که روی فضای هیلبرت عمل می کند بیان می شود. یعنی اگر حالت سیستم در زمان
$t_0$
،حالت 
$\pk$
باشد و در زمان
$t_1$
،حالت 
$| \psi ' \rangle$
باشد، آنگاه عملگر یکانی
$ U: \hi \rightarrow \hi$
وجود دارد به طوری که
$ U \pk = | \psi ' \rangle$

دقت کنید حالت سیستم در لحظه ی
$t_2$
باید با نرم واحد باشد که این خاصیت زمانی برقرار است که 
$U$
 طول را حفظ  کند.

این اصل در واقع فرمول بندي دیگري از معادله ی شرودینگر است. این معادله تحول زمانی یک سیستم کوانتمی را به صورت زیر بیان می کند:
\input{\formulaPATH{2}{8}}
که در آن
$H : \hi \rightarrow \hi$
یک عملگر هرمیتی (در حالت کلی دخواه) است که به آن همیلتونی گفته می شود.

می توان نشان داد
\cite{shankar}
که بین 
$U$
و 
$H$
رابطه ی زیر بر قرار است.
\input{\formulaPATH{2}{9}}


\subsubsection{اصل پنجم: سیستم های ترکیبی}
فضای هیلبرت متناظر با یک سیستم فیزیکی که متشکل از
$n$
سیستم کوچکتر (زیرسیستم) است از ضرب تانسوری فضاهای کوچک تر بدست می آید. به عبارت دیگر اگر فضای هیلبرت متناظر با سیستم
$i$
ام، 
$\hi_i$
باشد، فضای هیلبرت متناظر با کل
$n$
سیستم برابر است با:
\begin{equation*}
	\hi = \hi_1 \otimes \hi_2 \otimes \ldots \otimes \hi_n.
\end{equation*}

اگر سیستم 
$i$
ام در حالت
$ | \psi_i \rangle \in \hi$
باشد، کل سیستم در حالت
$ | \psi_1 \rangle \otimes | \psi_2 \rangle \otimes \ldots \otimes | \psi_n \rangle $
است.


%%%%%%%%%%%%%%%%%%%%%%%%%%%%%%%%%%%%%%%%%%%%%%%%%%%%%%%%%%%%%%%%%%%%%%%%%%%
