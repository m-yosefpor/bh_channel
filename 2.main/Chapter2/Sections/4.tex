\section{تئوری اطلاعات کوانتمی}
%%%%%%%%%%%%%%%%%%%%%%%%%%%%%%%%%%%%%%%%%%%%%%%%%%%%%%%%%%%%%%%%%%%%%%%%%%%

%%%%%%%%%%%%%%%%%%%%%%%%%%%%%%%%%%%%%%%%%%%%%%%%%%%%%%%%%%%%%%%%%%%%%%%%%%%
\subsection{عملگر چگالی، حالات مخلوط و خالص}
در قسمت قبل دیدیم که هر حالت سیستم به یک بردار در فضای هیلبرت متناظر است. 
مشابه مفهوم متغیر تصادفی، ما به ابزاری احتیاج داریم که بتوانیم با آن شرایطی که حالت سیستم کوانتمی مشخص نیست و فقط احتمال متناظر با آن حالات مشخص است را توصیف کنیم. ابزار ریاضی که چنین می کند، عملگر چگالی 
\LTRfootnote{Density Operator}
حالت است. وقتی سیستمی در چنین حالتی باشد، گفته می شود که در حالت مخلوط است. اگر بتوان حالت سیستم را با یک بردار در فضای احتمال مشخص کرد، چنین سیستمی در یک حالت خالص است.

فرض کنید سیستمی دارای حالت
$\pk$
در فضای 
$\hi$
باشد. لذا در حالت خالص است. می توان عملگر زیر را تعریف کرد:
\begin{equation}
		\pd = \pk \pb
\end{equation}
این یک عمگر در فضای
$\hi$
خواهد بود. حال اگر ترکیب خطی هر یک از این عملگر های چنینی را نیز به عنوان یک ماتریس چگالی در نظر بگیریم، هر یک از حالات به دست آمده یک حالت مخلوط خواهد بود. به عبارتی اگر حالات ممکن سیستم به صورت
$\{ |\alpha_1 \rangle, |\alpha_2 \rangle, \ldots , |\alpha_N \rangle \}$
با احتمال های
$\{p_1,p_2,\ldots p_N\}$
باشد، حالت مخلوط را می توان به صورت زیر نشان داد:
\begin{equation}
	\sum_{i=1} p_i |\alpha_i \rangle \langle \alpha_i  |
\end{equation}
اگر پایه ای در فضای هیلبرت در نظر گرفته شود، می توان نمایش ماتریسی از عملگر بالا داشت که به آن ماتریس چگالی
\LTRfootnote{Density Matrix}
گفته می شود.

حالت مخلوطی که در آن تمامی حالات ممکن دارای احتمال یکسان باشد، حالت مخلوط بیشینه 
\LTRfootnote{Maximally Mixed State}
خوانده می شود.

%%%%%%%%%%%%%%%%%%%%%%%%%%%%%%%%%%%%%%%%%%%%%%%%%%%%%%%%%%%%%%%%%%%%%%%%%%%
\subsection{اندازه گیری \lr{POVM}}
%%%%%%%%%%%%%%%%%%%%%%%%%%%%%%%%%%%%%%%%%%%%%%%%%%%%%%%%%%%%%%%%%%%%%%%%%%%
می توان اصل اندازه گیری در مکانیک کوانتمی را به صورت دیگری نیز بیان کرد. این بیان به اندازه گیری تصویری
\LTRfootnote{Projective Measurement}
یا انداره گیری فون نویمان 
\LTRfootnote{von Neumann Measurement}
معروف می باشد
\cite{chuang}
:

اندازه گیری بر روي یک سیستم با فضاي هیلبرت
$\hi$
با عملگر های خطی
\begin{equation*}
	M_1 , M_2, \ldots , M_k : \hi \rightarrow \hi,
\end{equation*}
مشخص می شود به طوری که 
\begin{equation}
\sum_{i=1}^{k} M_i^\dagger M_i = I_{\hi}
\end{equation}

که در آن 
$I_{\hi}$
عملگر همانی بر روی
$\hi$
است. اگر حالت سیستم
$\pk \in \hi$
باشد، حاصل اندازه گیری با احتمال
$p_i = \pb  M_i^\dagger M_i \pk$
برابر 
$i$
می شود. به علاوه اگر حاصل اندازه گیری
$1 \leq i \leq k$
باشد، حالت سیستم به 
\begin{equation}
| \psi ' \rangle = \frac{M_i \pk}{\| M_i \pk \|}
\end{equation}
فرو می پاشد.

همان طور که در بالا مشاهده می شود، عبارت احتمال 
$p_i$
به خود 
$M_i$
وابسته نیست، بلکه به 
$M_i^\dagger M_i$
وابسته است. به عبارت دیگر حتی اگر 
$M_i$
ها را نداشته باشیم اما 
$M_i^\dagger M_i$
را داشته باشیم، می توانیم تعیین کنیم که نتیجه ی آزمایش با چه احتمالی با برابر
$i$
است
\cite{ipm}
.

براي این منظور، زمانی که هدف صرفا تعیین احتمال نتیجه ي اندازه گیری باشد، راحت تر است که با عملگر های 
$E_i = M_i^\dagger M_i$
کار کنیم که به آن ها عملگر های 
\lr{POVM}
\LTRfootnote{Positive Operator Valued Measure}
گفته می شود.
دقت کنید که
$E_i ^\dagger = M_i^\dagger M_i = E_i$
بنابرین 
$E_i$
هرمیتی است. همچنین 
$E_i$
دارای دو شرط 
$E_i \geq 0$
و
$\sum E_i = I$
هستند.
%%%%%%%%%%%%%%%%%%%%%%%%%%%%%%%%%%%%%%%%%%%%%%%%%%%%%%%%%%%%%%%%%%%%%%%%%%%
\subsection{در هم تنیدگی}
همانطور که گفته شد، فضای هیلبرت یک سیستم ترکیبی، ضرب تانسوری زیر فضاهای آن است.
\begin{equation}
	\hi = \hi_A \otimes \hi_B
\end{equation}
فرض کنید که 
$\pk \in \hi$
.اگر بتوان
$\pk$
را به صورت 
$\pk_A \otimes \pk_B$
نوشت به طوری که 
$\pk_A \in \hi_A$
و
$\pk_B \in \hi_B$
آن گاه این حالت، جداپذیر 
\LTRfootnote{Separable State}
خوانده می شود. در غیر این صورت به آن یک حالت در هم تنیده
\LTRfootnote{Entangled State}
گفته می شود.
%%%%%%%%%%%%%%%%%%%%%%%%%%%%%%%%%%%%%%%%%%%%%%%%%%%%%%%%%%%%%%%%%%%%%%%%%%%
\subsection{آنتروپی فون نویمان}
مشابه تعریف آنتروپی شانون برای اطلاعات یک متغیر تصادفی کلاسیک، در انیجا به تعریف آنتروپی فون نویمان
\LTRfootnote{von Neumann Entropy}
 برای یک عملگر چگالی
$\rho^A$
 می پردازیم:
\input{\formulaPATH{2}{11}}
که در آن
$tr(.)$
نشان دهنده ی 
اثر
\LTRfootnote{Trace}
ماتریس می باشد و مطابق معمول
$\log$
در مبنای 2 می باشد
\cite{wilde, chuang}.

اگر 
$\rho$
یک حالت خالص باشد، آنتروپی آن صفر خواهد بود و برعکس. همچنین اگر بعد فضای هیلبرت 
$N$
باشد، آنتروپی بیشینه و برابر
$\log(N)$
خواهد بود اگر و فقط اگر 
$\rho$
یک حالت بیشینه مخلوط باشد.
%%%%%%%%%%%%%%%%%%%%%%%%%%%%%%%%%%%%%%%%%%%%%%%%%%%%%%%%%%%%%%%%%%%%%%%%%%%
\subsection{آنتروپی مشترک و آنتروپی شرطی}

فرض کنید که دو سیستم 
$A$
و
$B$
داشته باشیم. یک حالت از این سیستم به صورت
$\rho^{AB} \in \hi_{AB}$
در نظر بگیرید، به طوریکه:
 $\hi_{AB} = \hi_A \otimes \hi_B$

آنتروپی مشترک این دو به صورت زیر تعریف می شود:
\input{\formulaPATH{2}{15}}

تعریف می کنیم:
\begin{align}
	\rho^{A}   &= tr_B(\rho^{AB}) \\
	\rho^{B}   &= tr_A (\rho^{AB}) 
\end{align}

که از اثر جزئی
\LTRfootnote{Partial Trace}
در تعاریف بالا استفاده شده است. این عبارات مشابه توزیع های حاشیه ای در تئوری اطلاعات کلاسیک می باشند.

آنتروپی شرطی به صورت زیر تعریف می شود:

\input{\formulaPATH{2}{13}}
%%%%%%%%%%%%%%%%%%%%%%%%%%%%%%%%%%%%%%%%%%%%%%%%%%%%%%%%%%%%%%%%%%%%%%%%%%%
\subsection{آنتروپی نسبی کوانتمی و اطلاعات متقابل}
همانطور که در قسمت تئوری اطلاعات کلاسیک آنتروپی نسبی تعریف کردیم، در اینجا هم بسیار پرکابرد است که مفهومی معادل آن در حالت کوانتمی تعریف کنیم. آنتروپی نسبی
$\rho$
به
$\sigma$
به صورت زیر تعریف می شود:
\input{\formulaPATH{2}{12}}

مشابه حالت کلاسیک، اطلاعات متقابل به صورت زیر تعریف می شود:
\input{\formulaPATH{2}{14}}
%%%%%%%%%%%%%%%%%%%%%%%%%%%%%%%%%%%%%%%%%%%%%%%%%%%%%%%%%%%%%%%%%%%%%%%%%%%
\subsection{کانال کوانتمی}
در تئوری اطلاعات کوانتمی، کانال های کلاسیک (متشکل از یک توزیع شرطی
$p(y|x)$
)
جای خود را به کانال های کوانتمی می دهند. کانال کوانتمی نگاشتی خطی است که یک ماتریس چگالی (ماتریسی مثبت نیمه معین با اثر یک) را به ماتریس هاي چگالی می برد و به آن نگاشت خطی کاملا مثبت و حافظ اثر یا
$CPTP$
\LTRfootnote{Completely Positive Trace-Preserving Map}
 نیز گفته می شود.
 کانال میان فرستنده و گیرنده ممکن است کانالی با ورودي کلاسیک و خروجی کلاسیک (اصطلاحا کانال 
 $cc$
 یا کلاسیک-کلاسیک)، کانالی با ورودي کلاسیک و خروجی کوانتمی(اصطلاحا کانال 
 $cq$
 )،کانالی با ورودي کوانتمی و خروجی کلاسیک ((اصطلاحا کانال
 $qc$
 )، و یا کانالی با ورودي کوانتمی و خروجی کوانتمی (اصطلاحا کانال 
 $qq$
 )
 باشد
 \cite{ipm}.
%%%%%%%%%%%%%%%%%%%%%%%%%%%%%%%%%%%%%%%%%%%%%%%%%%%%%%%%%%%%%%%%%%%%%%%%%%%

\subsection{ اطلاعات منسجم  }

در این بخش به تعاریف جدیدی می پردازیم که برای قسمت های بعد در تعریف ظرفیت کانال کوانتمی به آن ها نیاز پیدا خواهیم کرد.

اطلاعات منسجم 
\LTRfootnote{Coherent Information}
یک حالت
$\rho ^{AB}$
به صورت زیر تعریف می شود:
\input{\formulaPATH{2}{16}}
می توان نشان داد که اگر محیط 
$E$
یک خالص ساز
\LTRfootnote{Purification}
برای این حالت باشد، آن گاه می توان اطلاعات منسجم را به صورت زیر نوشت
\cite{wilde}
:
\input{\formulaPATH{2}{17}} 




%%%%%%%%%%%%%%%%%%%%%%%%%%%%%%%%%%%%%%%%%%%%%%%%%%%%%%%%%%%%%%%%%%%%%%%%%%%
\subsection{ظرفیت کلاسیکی یک کانال کوانتمی (ظرفیت \lr{Holevo})}
حال می توان قضیه ی زیر را که در مورد ظرفیت کلاسیکی یک کانال کوانتمی است و به قضیه ی
\lr{HSW}
\LTRfootnote{Holevo–Schumacher–Westmoreland}
معروف است، بیان کرد.

ظرفیت کلاسیکی یک کانال کوانتمی، سوپریمم تمام نرخ های قابل حصول می باشد :
\input{\formulaPATH{2}{18}}
که در آن ماکسیمم گیری روی تمام حالات 
$\rho^{XB}$
که به فرم زیر هستند انجام می شود
\cite{wilde,bosonic}
:
\input{\formulaPATH{2}{19}}

می توان نشان داد
\cite{wilde}
در حالتی که حالت سیستم به صورت بالا توصیف شود، اطلاعات 
\lr{Holevo}
به عبارت زیر کاهش پیدا می کند:
\begin{equation} \label{ccshort}
	\chi(\cn) = \max_\rho I(X;B)_\rho
\end{equation}

در حالت کلی، محاسبه ی ظرفیت کلاسیک هر کانال کوانتمی دلخواه مساله ای ناکارآمد زمانی
\LTRfootnote{intractable}
است. به این خاطر که فرمول ظرفیت چند حرفی 
\LTRfootnote{Multi-letter}
می باشد. اما در موارد خاصی می توان نشان داد که رابطه ی
$\chi_{reg} (\cn) = \chi(\cn)$
برقرار است و در این حالت که گفته می شود فرمول ظرفیت تک حرفی
\LTRfootnote{Single-letter}
است، می توان ظرفیت را محاسبه کرد
\cite{bosonic}.
%%%%%%%%%%%%%%%%%%%%%%%%%%%%%%%%%%%%%%%%%%%%%%%%%%%%%%%%%%%%%%%%%%%%%%%%%%%
\subsection{ظرفیت کوانتمی}
ظرفیت کوانتمی کانال
$\cn^{A \to B}$
 سوپریمم روی تمام نرخ های قابل دستیابی برای مخابره ی کوانتمی است
 \cite{wilde}
 :

\begin{align} \label{qdef}
	Q(\cn) &= \lim_{n \to \infty} \frac{1}{n} Q^{(1)} (\cn^{\otimes n}) \\
	Q^{(1)} (\cn) &= \max_{\rho^A} I_{coh} (\cn, \rho^A)
\end{align}


 که در آن ماکسیمم گیری روی تمام عملگرهای چگالی ورودی انجام می شود.
 
 مشابه حالت قبل،  محاسبه ی ظرفیت کلاسیک هر کانال کوانتمی دلخواه مساله ای ناکارآمد زمانی
 است. به این خاطر که فرمول ظرفیت چند حرفی است،
  اما در موارد خاصی در اینجا نیز می توان نشان داد که
 $Q (\cn) = Q^{(1)}(\cn)$
 و لذا می توان ظرفیت را محاسبه نمود
 \cite{qit,bosonic}.
%%%%%%%%%%%%%%%%%%%%%%%%%%%%%%%%%%%%%%%%%%%%%%%%%%%%%%%%%%%%%%%%%%%%%%%%%%%
\subsection{وفاداری}

یک راه برای اندازه گیری نزدیکی دو حالت استفاده از وفاداری
\LTRfootnote{Fidelity}
 می باشد. وفاداری اولمان
\LTRfootnote{Uhlmann}
بین دو حالت مخلوط
$\rho^A$
و
$\sigma^A$
حداکثر هم پوشانی بین خالص ساز های متناظرشان است، به طوری که ماکسیمم سازی روی تمام عملگر های یکانی
$U$
روی سیستم خالص ساز
$R$
است.
\input{\formulaPATH{2}{21}}
در معادله ی بالا، 
$\phi_\rho$
و
$\phi_\sigma$
هر یک خالص ساز متناظر با حالات مخلوط
$\rho$
و
$\sigma$
می باشد.

می توان نشان داد
\cite{wilde}
که این وفاداری را می توان به شکل نرم 
$l1$
زیر هم نوشت.
\input{\formulaPATH{2}{22}}
%%%%%%%%%%%%%%%%%%%%%%%%%%%%%%%%%%%%%%%%%%%%%%%%%%%%%%%%%%%%%%%%%%%%%%%%%%%
\subsection{ تعمیم ایزومتری کانال و متمم کانال}

یک کانال ایزومتری
\LTRfootnote{Isometric Channel}
$V^{A' \to B}$
کانالی است که بتوان آن را به صورت
$V(\rho) = U \rho U^\dagger$
نوشت به طوریکه 
$U^{A \to B}$
یک ایزومتری باشد.

به ازای هر کانال
$\cn^{A' \to B}$
همیشه کانالی مانند
$V^{A' \to BE}$
وجود دارد 
\cite{broadcast}
به طوریکه 
\begin{equation}
	\cn^{A' \to B} = tr_E V^{A' \to BE}
\end{equation}
به این کانال، تعمیم ایزومتری
\LTRfootnote{Isometric Extension}
 کانال 
$\cn^{A' \to B}$
گفته می شود.

با توجه به تعریف بالا، متمم کانال
$\cn^{A' \to B}$
را به صورت زیر تعریف می کنیم:
\begin{equation}
\cn^{A' \to B} = tr_B V^{A' \to BE}
\end{equation}

%%%%%%%%%%%%%%%%%%%%%%%%%%%%%%%%%%%%%%%%%%%%%%%%%%%%%%%%%%%%%%%%%%%%%%%%%%%