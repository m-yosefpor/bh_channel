\section{ظرفیت کلاسیکی سیاه چاله (ظرفیت \lr{Holevo})}
%%%%%%%%%%%%%%%%%%%%%%%%%%%%%%%%%%%%%%%%%%%%%%%%%%%%%%%%%%%%%%%%%%%%%%%
\subsection{مدل کانال سیاه چاله}
		در این بررسی، یک مدل شبه کلاسیک  (تقریبی که میدان های ماده را به صورت کوانتمی و میدان های گرانشی را به صورت کلاسیک در نظر می گیرد) برای سیاه چاله در نظر میگیریم بدین معنی که حالت ماکروسکوپیک شوارتزشیلد در نظر گرفته می شود. علاوه بر این، از اثر واکنش برگشتی
\LTRfootnote{Backreaction}
(که اثر سیاه چاله بر روی متریک فضای اطرافش است) و همچنین اثر پس زدن
\LTRfootnote{Recoil}
که پایستگی تکانه است، صرف نظر می شود. به علاوه، همان طور که مرسوم است، فرض میکنیم که سیاه چاله یک میدان بدون جرم تابش کند و از انتقال به سرخ گرانشی
\LTRfootnote{Gravitational Redshift}
صرف نظر می شود. البته با اینکه چنین فرض هایی انجام می دهیم، انتظار داریم که اگر این فرض ها را نیز لحاظ نکنیم، باز نتیجه از نظر کیفی تغییری نکند
\cite{qit}.

تحول زمانی این نوع سیاه چاله توسط سورکین
\cite{sorkin}
بررسی شده است. در این بررسی هامیلتونی سیاه چاله به صورت زیر به دست آمده است:
\begin{equation}
	H_k =  i g_k (a_k^\dagger b_{-k}^\dagger - a_k b_{-k}) + i g_k ' (a_k^\dagger c_k - a_k c_k^\dagger)
\end{equation}


\begin{equation}
	H = \sum_{k=-\infty}^{\infty} H_k = \sum_{-\infty}^{\infty} i g_k (a_k^\dagger b_{-k}^\dagger - a_k b_{-k}) + i g_k ' (a_k^\dagger c_k - a_k c_k^\dagger)
\end{equation}

که در آن
$a_k^\dagger$
،
$a_k$
به ترتیب اپراتور های خلق و فنای مود 
$k$
ام ذره در فضای داخل،
و
$b_k^\dagger$
،
$b_k$
به ترتیب اپراتور های خلق و فنای مود 
$k$
ام ذره در فضای خارج است. به همین ترتیب
$a_{-k}$
عملگر فنای یک پادذره مود 
$k$
می باشد و مشابها برای 
$a_{-k}$
و
$b_{-k}$
.
$c_k$
نیز به مود های دیر هنگام مربوط می باشد.



همچنین می توان 
$A_k$
که خلا خارج را فنا می دهد و 
$B_k$
که خلا داخل را فنا می دهد را
 به ترتیب به صورت زیر نوشت:
\begin{equation}
	A_k = e^{-i H} a_k e^{i H}
\end{equation}

\begin{equation}
B_k = e^{-i H} b_k e^{i H}
\end{equation}

با استفاده از لم بیکر-کمپبل-هاسدورف 
\LTRfootnote{Baker-Campbell-Hausdorff}
می توانیم
$A_k$
را به صورت زیر بازنویسی کنیم:
\begin{equation} \label{param}
	A_k = \alpha a_k - \beta b_{-k}^\dagger + \gamma c_k
\end{equation}
که در آن
\begin{align} \label{parr}
	\alpha ^2 &= \cos ^2(g_k ' \omega) \\
	\beta ^2 & = (\frac{g_k}{g_k '})^2 \frac{\sin ^2 (g_k ' \omega)}{\omega^2}  \\
	\gamma^2 & = \frac{\sin ^2 (g_k ' \omega)}{\omega^2}
\end{align}

و
\begin{equation}
	\omega ^2 = 1 - (\frac{g_k}{g_k '})^2
\end{equation}

لذا با توجه به بالا می توان نتیجه گرفت:
\begin{equation} \label{paran}
	\alpha_k ^2 - \beta_k ^2 + \gamma_k^2 = 1
\end{equation}


حال می خواهیم مدل سیاه چاله را در نظر بگیریم. برای این کار دو حالت ممکن وجود دارد. یکی اینکه به مود های زود هنگام 
\LTRfootnote{Early-Time}
توجه کنیم. به عبارتی، حالتی که قبل از تشکیل سیاه چاله آماده شده است را در نظر بگیریم و سر نوشت اطلاعات موجود در آن را پس از فروپاشی به سیاه چاله بررسی کنیم.  در این صورت ورودی کانال  همان حالت آماده شده می باشد و خروجی کانال را می توان تابش هاوکینگ یا تابش بر انگیخته فرض کرد. لذا می توان با محاسبه ی ظرفیت کانال بررسی کرد که آیا اطلاعات موجود در حالت آماده شده، در هر یک از تابش ها وجود دارد یا خیر. اگر ظرفیت مثبت باشد، یعنی اطلاعات را می توان از خروجی مورد نظر (یعنی تابش هاوکینگ یا تابش برانگیخته) بازیابی کرد.

یک حالت دیگر، حالت دیر هنگام 
\LTRfootnote{Late-Time}
می باشد و بدین صورت است که پس از تشکیل سیاه چاله جسمی به داخل آن فرستاده شود. همانطور که در شکل 
\ref{fig331}
نشان داده می شود،
 می توان ورودی کانال را جسم فرستاده شده به داخل سیاه چاله فرض کرد و  خروجی کانال را هر یک از فلش های موجود در شکل گرفت. هر کدام از این فرض ها، یک کانال متناظر ایحاد میکنند که می توان سرنوشت اطلاعات را آنجا جست و جو کرد. به عنوان مثال خروجی کانال می تواند هر یک از تابش های برانگیخته داخل یا خارج باشد. همینطور می تواند تابش هاوکینگ داخل یا تابش هاوکینگ خارج باشد. البته در ادامه خواهیم دید که برای تابش هاوکینگ در نظر گرفتن یکی از تابش ها (داخل یا خارج) کفایت می کند.
\begin{figure}
	\centering
	\includegraphics[width=\linewidth]{\pngPATH{3}{1}}
	\caption{در حالت دیر هنگام تابش های حاصل از سیاه چاله نشان داده شده است. \cite{cit}}
	\label{fig331}
\end{figure}
%%%%%%%%%%%%%%%%%%%%%%%%%%%%%%%%%%%%%%%%%%%%%%%%%%%%%%%%%%%%%%%%%%%%%%%
\subsection{بررسی ظرفیت تابش هاوکینگ}

در ابتدا، فرض کنید که مود های زود هنگام مورد نظر است و میخواهیم ظرفیت کلاسیکی تابش هاوکینگ را بررسی کنیم. 


در این حالت می توان ایزومتری کانال را  به صورت زیر به دست آورد
\cite{cit}
:

\begin{equation} \label{isomw}
V_\omega = exp(-i H_s) = \prod_{k=-\infty}^{\infty} e^{ g_k (a_k^\dagger b_{-k}^\dagger - a_k b_{-k}) +  g_k ' (a_k^\dagger c_k - a_k c_k^\dagger)}
\end{equation}


همچنین داریم:
\begin{equation} \label{vout}
	|vac\rangle_{out} = V_\omega |vac\rangle
\end{equation} 

لذا ماتریس چگالی تابش خروجی به صورت زیر به دست می آید:
\begin{equation} \label{rhoi}
	\rho^B = tr_{A} |vac\rangle_{out} \langle vac | = \prod_k \rho_k \otimes \rho_{-k}
\end{equation}
که 
$\rho_k$
و
$\rho_{-k}$
به ترتیب ماتریس چگالی برای ذرات و پادذرات می باشند.

با توجه به اینکه مود های زود هنگام مورد بررسی است لذا 
$\gamma = 0$
و از
رابطه ی
\ref{param}
داریم:
\begin{equation}
	\alpha_k^2 -\beta_k^2 = 1
\end{equation} 

همچنین چون جواب باید تحلیلی باشد، طبق استدلال ها آورده شده در 
\cite{hr}
رابطه زیر باید برقرار باشد:
\begin{equation}
	\alpha_k ^2= e^{\omega / T} \beta_k ^2
\end{equation} 
 که در آن
 $\omega = |k|$
 و 
 $T$
 دمای سیاه چاله است.
 
لذا به دست می آوریم:
\begin{align}
	\beta_k^2 &= \frac{e^{-\omega / T}}{1- e^{-\omega / T}} \\
	\alpha_k^2 &= \frac{1}{1- e^{-\omega / T}}
\end{align}
 
با توجه به این نتیجه و رابطه ی 
 \ref{rhoi}
 و
 \ref{vout}
 و جایگذاری از رابطه ی
 \ref{isomw}
 داریم:
 \begin{equation}
 	S(p_k) = -tr \rho_k \log \rho_k = \frac{\omega / T}{e^{\omega / T} -1 } + \log (1-e^{-\omega / T})
 \end{equation}
 
 مشاهده می شود که آنتروپی کاملا گرمایی است، یعنی به جسم تشکیل دهنده سیاه چاله هیچ ارتباطی ندارد و لذا ظرفیت آن صفر است.
 
 تعداد ذراتی که تابیده می شوند برابر است با:
 \begin{equation}
 	\langle N_1 \rangle = \sum_{k=-\infty}^{\infty} \ _{out} \langle 0 |a_k^\dagger a_k |0\rangle _{out} = \sum_{k=-\infty}^{\infty} \beta_k^2
 \end{equation}
 که 
 $\ _{out} \langle 0|$
 نمایانگر ترانهاده و مزدوج 
 $|0\rangle_{out}$
 است.
%%%%%%%%%%%%%%%%%%%%%%%%%%%%%%%%%%%%%%%%%%%%%%%%%%%%%%%%%%%%%%%%%%%%%%%
\subsection{بررسی ظرفیت تابش برانگیخته برای مود های زودهنگام}
حال تابش های برانگیخته را بررسی می کنیم. برای این کار حالت های خروجی را به صورت
$\pk_{out}$
نشان می دهیم. همچنین فرض کنید 
$m$
ذره در ابتدا قبل از تشکیل سیاه چاله به آن وارد شده اند. لذا داریم:
\begin{equation}
	\pk_{out} = V_\omega |m\rangle_a |0\rangle_b
\end{equation}

در این حالت ماتریس چگالی خروجی برای ناحیه خارج سیاه چاله را به دست می آوریم:
\begin{equation}
	\rho^A = tr_B \pk_{out} \pb = \rho_{k|m} \otimes \rho_{-k|0}
\end{equation}
که در آن
$\rho_{k|m}$
ماتریس چگالی مود 
$k$
به شرطی که 
$m$
ذره با مود
$k$
برخورد کنند و 
$\rho_{-k|0}$
ماتریس چگالی پادذره ها با مود 
$k$
اگر هیچ پادذره ای با مود
$k$
برخورد نکند، می باشد.

لذا اگر

\begin{equation}
	\rho_{k|m} = \sum_{n=0}^{\infty} p(n|m) |n\rangle \langle n|
\end{equation}

آنگاه داریم
\cite{cit}
:
\begin{equation} \label{csharti}
	p(n|m) = (1-z) ^{(m+1)} z^n \binom{m+n}{m}
\end{equation}

که در آن
\begin{equation}
	z = \frac{\beta^2}{1+\beta^2} = e^{-\omega/T}
\end{equation}

در نتیجه میانگین تعداد ذرات تابش شده به ناحیه ی خارج برابر است با:
\begin{align}
	\langle N_1 \rangle &= \sum_{k=-\infty}^{\infty} \ _{out} \langle 0 |a_k^\dagger a_k |0\rangle _{out} \\
	&= (1+\beta^2)m + \beta^2
\end{align}

اگر سیستم آماده شده توسط ارسال کننده 
$X$
باشد، لذا اطلاعات متقابل بین تابش و 
$X$
داریم:
\begin{equation} \label{cccc}
	I(X;A)_\rho = S(\rho^A) + S(\rho^X) - S(\rho^{AX})
\end{equation}

لذا طبق رابطه ی 
\ref{ccshort}
داریم:
\begin{equation}
\chi(\cn) = \max_\rho I(X;A)_\rho
\end{equation}

با فرض اینکه آماده کننده اطلاعات را در حالت دوخطی
\LTRfootnote{Dualrail}
کد کرده است، به طوری که حالت 
$1$
 احتمال
$p$
را دارد و حالت 
$0$
احتمال
$q=1-p$
را داشته باشد:
\begin{equation}
	S(\rho^X) = -p \log p - (1-p) \log (1-p)
\end{equation}

از رابطه ی
\ref{csharti}
می توان نشان داد که 
\cite{cit}
:
\begin{equation}
	S(\rho^{AX}) = S(\rho^X) - 3 \log (1-z) - \frac{3z}{1-z} \log z - (1-z)^2 \Delta
\end{equation}

که 
\begin{equation}
	\Delta = \sum_{m=0}^{\infty} z^m (m+1) \log(m+1)
\end{equation}

و همچنین:
\begin{equation}
	\rho^A (p) = (1-z)^3 \times \sum_{m,m' =0}^{\infty} z^{m+m'-1} (pm' + (1-p)m) |mm'\rangle \langle mm'|
\end{equation}

که با توجه به تقارن نسبت به 
$p \to 1-p$
می توان نشان داد که در 
$p=1/2$
آنتروپی ماکسیمم می شود:

\begin{align}
	S(\rho^A(p=1/2)) &= 1- 3(\log(1-z) + \frac{z}{1-z} \log z) \\
	& - (1-z)^3 \sum_{m=0}^{\infty} z^m(m+1)(m+2) \log(m+1) 
\end{align}

پس با توجه به رابطه
\ref{cccc}
به دست می آوریم که ظرفیت آن به صورت زیر است:
\begin{align} \label{ccfin}
	\chi &= 1 - \frac{1}{2} (1-z)^3 \sum_{m=0}^{\infty}  z^m (m+1)(m+2) \log (m+1) \\
	& + (1-z)^2 \sum_{m=0}^{\infty} z^m (m+1) \log (m+1)
\end{align}


این همان ظرفیت 
\lr{Holevo}
کانال
\lr{Unruh}
می باشد.
%%%%%%%%%%%%%%%%%%%%%%%%%%%%%%%%%%%%%%%%%%%%%%%%%%%%%%%%%%%%%%%%%%%%%%%
\subsection{بررسی ظرفیت تابش برانگیخته برای مود های دیرهنگام} %halat khas??

در این قسمت می خواهیم بررسی کنیم که اگر ذره ای پس از شکل گیری سیاه چاله به آن برخورد کند چه سرنوشتی برای اطلاعات موجود در آن در انتظارش است. برای این کار، باید مود های دیر هنگام را در نظر بگیریم. لذا در این قسمت
$\gamma \neq 0$
می باشد. می توان نشان داد
\cite{cit}
که :
\begin{align}
	\beta_k^2 &= \frac{\Gamma}{e^{\omega / T} - 1} \\
	\alpha_k^2 &= \frac{\Gamma}{1- e^{-\omega / T}}
\end{align}

که در آن 
$\Gamma = 1 - \gamma ^2$
مقدار جذب 
\LTRfootnote{Absorptiviy}
سیاه چاله می باشد.

همچنین داریم:
\begin{equation}
	\pk_{out} = V_\omega |m\rangle_{in}
\end{equation}


\begin{equation} \label{rhoic}
\rho^B = tr_{A} |vac\rangle_{out} \langle vac |
\end{equation}

تعداد ذراتی که در این تابش ها هستند برابر است با:
 \begin{equation}
 \langle N_1 \rangle = \sum_{k=-\infty}^{\infty} \ _{out} \langle 0 |a_k^\dagger a_k |0\rangle _{out} =  \beta^2 + \gamma^2 m
 \end{equation}
مشاهده می شود علاوه بر 
$\beta^2$
ذره تابش شده به خاطر تابش هاوکینگ، 
$\gamma^2 m$
ذره دیگر تابش می شود که 
$(1-\alpha^2)m$
تای آن به خاطر پراکندگی الاستیک با احتمال جذب کوانتمی 
$\alpha^2$
و 
$\beta^2 m $
ذره به خاطر تابش بر انگیخته می باشد
\cite{cit}
.

در این حالت ماتریس چگالی خروجی برای ناحیه خارج سیاه چاله را به دست می آوریم:
\begin{equation}
\rho^A = tr_B \pk_{out} \pb = \rho_{k|m} \otimes \rho_{-k|0}
\end{equation}

اگر

\begin{equation}
\rho_{k|m} = \sum_{n=0}^{\infty} p(n|m) |n\rangle \langle n|
\end{equation}

آنگاه داریم
\cite{cit}
:
\begin{equation}
p(n|m) = R_{nm} \sum_{k=0}^{min(n,m)} (-1)^k \binom{m}{k} \binom{m+n-k}{n-k} (1-\frac{\gamma^2}{\alpha^2 \beta^2})^k
\end{equation}

که در آن
\begin{equation}
	R_{nm} = \frac{1}{1+\beta^2} (\frac{\beta^2}{1+\beta^2})^{m+n} (\frac{\alpha^2}{\beta^2})^m
\end{equation}

از طرفی می توان نتیجه بگیریم که 
\begin{equation}
	(g'_k/g_k)^2 = 1 + \frac{1-\alpha^2}{\alpha^2} e^{\omega/T}
\end{equation}
که با توجه به اینکه 
$0 \leq \alpha \leq 1$
لذا نتیجه می گیریم که
$g'_k \geq g_k$
یا 
$g'_k = 0$

اگر 
$g'_k = 0$
، این حالت متناسب با همان تابش هاوکینگ می باشد
\cite{cit}
که آن را قبلا بررسی کردیم.

برای محاسبه ظرفیت در حالت
$g'_k \geq g_k$
 یک حالت خاص را مد نظر قرار می دهیم که  
$g'_k = g_k$.
در این حالت با توجه به روابط 
\ref{parr}
،
$\alpha^2 =1$
می باشد و به عبارتی این حالت حدی سیاه چاله کاملا بازتاب کننده می باشد. در این حالت داریم:

\begin{equation}
p(n|m) = \frac{e^{-n\omega/T}}{(1+e^{-\omega/T})^{n+m+1}} \binom{m+n}{m}
\end{equation}

\begin{equation}
	p(n|m) = (\frac{z}{1+z})^n (1- \frac{z}{1+z})^{m+1} \binom{m+n}{m}
\end{equation}

مشاهده می کنیم که این دقیقا مشابه مود زودهنگام می باشد با این تفاوت که 
$z \to \frac{z}{1+z}$

لذا ظرفیت این نیز دقیقا مشابه معادله ی 
\ref{ccfin}
است وقتی  جایگذاری
$z \to \frac{z}{1+z}$
را انجام دهیم. پس داریم:

\begin{align}
	\chi &= 1 - \frac{1}{2} (1-\zeta)^3 \sum_{m=0}^{\infty}  \zeta^m (m+1)(m+2) \log (m+1) \\
	& + (1-\zeta)^2 \sum_{m=0}^{\infty} \zeta^m (m+1) \log (m+1)
\end{align}

که در آن
\begin{equation}
	\zeta = \frac{z}{1+z}
\end{equation}

%%%%%%%%%%%%%%%%%%%%%%%%%%%%%%%%%%%%%%%%%%%%%%%%%%%%%%%%%%%%
\subsection{بررسی نتایج}
در این قسمت ظرفیت های کلاسیکی کانال کوانتمی سیاه چاله را در حالت های مختلف مورد بررسی قرار دادیم و مقدار اطلاعات موجود در تابش های هاوکینگ و بر انگیخته را در مود های زودهنگام، یعنی وقتی ورودی حالت آماده شده قبل از تشکیل سیاه چاله باشد، و مود های دیر هنگام ، یعنی وقتی که پس از تشکیل سیاه چاله به آن جسمی را ارسال کنیم مورد بررسی قرار دادیم.

همانطور که ذکر شد، این موضوع نتیجه می شود که آنتروپی خروجی تابش هاوکینگ هیچ ارتباطی با ورودی کانال ندارد و کاملا گرمایی است و نتیجه گرفته شد که ظرفیت آن صفر است. در حالت بعدی، تابش بر انگیخته را در مود های زود هنگام بررسی کردیم و به دست آوردیم که ظرفیت آن طبق رابطه ی زیر می باشد:
\begin{align}
	\chi &= 1 - \frac{1}{2} (1-z)^3 \sum_{m=0}^{\infty}  z^m (m+1)(m+2) \log (m+1) \\
	& + (1-z)^2 \sum_{m=0}^{\infty} z^m (m+1) \log (m+1)
\end{align}


که با توجه به نمودار رسم شده در شکل
\ref{fig333}
مشاهده می شود که در حالتی که 
$z=0$
یعنی وقتی دمای سیاه چاله
$T \to 0$
اطلاعات را میتوان با همان نرخی که تولید می شود بازبابی کرد. با افزایش دمای سیاه چاله،
$z$
افزایش می یابد تا به 
$z=1$
می رسد. حتی در این حالت حدی نیز ظرفیت صفر نمی باشد و طبق معادله 
\ref{ccfin}
به یک مقدار مثبت میل می کند.

در حالت دیر هنگام به این نتیجه رسیدیم که در صورتی که
 $g'_k = g_k$
 آنگاه ظرفیت کانال به صورت 
 
 \begin{align}
 	\chi &= 1 - \frac{1}{2} (1-\zeta)^3 \sum_{m=0}^{\infty}  \zeta^m (m+1)(m+2) \log (m+1) \\
 	& + (1-\zeta)^2 \sum_{m=0}^{\infty} \zeta^m (m+1) \log (m+1)
 \end{align}
 
 \begin{figure}
 	\centering
 	\includegraphics[width=0.67\linewidth]{\pngPATH{3}{3}}
 	\caption{نمودار ظرفیت \lr{Holevo} تابش برانگیخته در مود های زود هنگام  (ظرفیت کانال \lr{Unruh}) بر حسب پارامتر $z$ \cite{cit}}
 	\label{fig333}
 \end{figure}
 
 است که
 \begin{equation}
 \zeta = \frac{z}{(1+z)}
 \end{equation}
 
 لذا نتیجه گرفته می شود که در این حالت نیز ظرفیت سیاه چاله همواره مثبت می باشد و لذا می توان اطلاعات را در تابش برانگیخته یافت و آن را بازیابی کرد.