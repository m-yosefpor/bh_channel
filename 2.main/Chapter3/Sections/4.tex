\section{محاسبه ی وفاداری سیاه چاله}
%%%%%%%%%%%%%%%%%%%%%%%%%%%%%%%%%%%%%%%%%%%%%%%%%%%%%%%%%%%%%%%%%%%%%%%%
\subsection{سیاه چاله به عنوان یک \lr{cloner}}
همان طور که در بخش های قبل گفته شد، اگر پس از تشکیل سیاه چاله جسمی به درون آن پرتاب شود، کلون هایی از آن به بیرون بازتاب می شود و پادکلون هایی به درون فرستاده می شوند. منظور از کلون در اینجا یک کپی عینی از اطلاعات سیستم نمی باشد. چون با توجه به قضیه ی 
\lr{no-cloning}
اصلا چنین چیزی ممکن نمی باشد و اگر سیاه چاله یک کلون واقعی از اطلاعات ایجاد می کرد، این خود منجر به پارادوکس می شد. اثبات می شود که وفاداری یک کلونر 
$N \to M$
 در بهترین حالت، که به وفاداری بهینه 
\LTRfootnote{Optimal Fidelity}
مشهور است،  برابر مقدار زیر است
\cite{cloner}
:
\begin{equation} \label{fidopt}
	F_{opt} = \frac{M(N+1)+N}{M(N+2)}
\end{equation}

در این قسمت نشان می دهیم که تابش بی ویژگی هاوکینگ به عنوان نویزی عمل می کند که این از کلون دقیق جلوگیری می کند و فقط کلون های تقریبی ایجاد می شود. برای این که مشخص کنیم کلون های ایجاد شده چه مقدار به اطلاعات اصلی شبیه هستند، وفاداری بین این حالت ها را محاسبه می کنیم.

%%%%%%%%%%%%%%%%%%%%%%%%%%%%%%%%%%%%%%%%%%%%%%%%%%%%%%%%%%%%%%%%%%%%%%%%
\subsection{محاسبه ی وفاداری سیاه چاله}
در اینجا نیز مجددا مدل 
\lr{Bogoliubov}
سیاه چاله را در نظر میگیریم:
\begin{equation}
H_k =  i g_k (a_k^\dagger b_{-k}^\dagger - a_k b_{-k}) + i g_k ' (a_k^\dagger c_k - a_k c_k^\dagger)
\end{equation}

لذا به دست می آوریم:

\begin{equation}
A_k = e^{-i H} a_k e^{i H} = \alpha a_k - \beta b_{-k}^\dagger + \gamma c_k
\end{equation}

که 
\begin{align}
\alpha ^2 &= \cos ^2(g_k ' \omega) = \Gamma_0 = \frac{\Gamma}{1-e^{-\omega/T}} \\
\beta ^2 & = (\frac{g_k}{g_k '})^2 \frac{\sin ^2 (g_k ' \omega)}{\omega^2} = \frac{\Gamma}{e^{\omega/T}-1}  \\
\gamma^2 & = \frac{\sin ^2 (g_k ' \omega)}{\omega^2} = 1 - \Gamma
\end{align}

تعریف می کنیم:
\begin{equation}
	\xi = \frac{\gamma^2}{\alpha^2 \beta^2}
\end{equation}

پس برای ماتریس چگالی داریم
\cite{cloner}:

\begin{equation}
	\rho^A = \rho_{k|1} \otimes \rho_{-k|0}
\end{equation}


که
\begin{align}
	\rho_{k|1} &= \sum_m p(m|1) |m\rangle \langle m| \\
	& = \frac{\alpha^2}{(1+\beta^2)^2} \sum_{m=0}^{\infty} (\frac{\beta^2}{1+\beta^2})^m (1+m\xi) |m\rangle \langle m| \\
	\rho_{k|0} &= \frac{1}{1+\beta^2} \sum_{m=0}^{\infty} (\frac{\beta^2}{1+\beta^2})^m |m\rangle \langle m|
\end{align}

لذا طبق روابط بالا وفاداری
$1 \to M$
طبق زیر به دست می آید
\cite{cloner}:

\begin{equation} \label{fid1m}
	F_{1 \to M} = \frac{3+\xi + 2\xi M}{3(2+\xi M)}
\end{equation}


%%%%%%%%%%%%%%%%%%%%%%%%%%%%%%%%%%%%%%%%%%%%%%%%%%%%%%%%%%%%%%%%%%%%%%%%
\subsection{بررسی در حالت خاص کاملا بازتاب کننده}

اگر سیاه چاله کاملا بازتاب کننده باشد، 
$\Gamma_0 = \alpha^2 \to 0$
لذا 
$\xi \to \infty$
پس طبق رابطه ی 
\ref{fid1m}
با محاسبه ی حد داریم:
\begin{equation}
	\lim_{\xi \to \infty} F_{1 \to M} = \frac{2}{3} + \frac{1}{3M}
\end{equation}

مشاهده می شود که این دقیقا وفاداری بهینه در رابطه ی 
\ref{fidopt}
می باشد.

\subsection{بررسی در حالت خاص کاملا جذب کننده}

حالت خاص دیگر، حالت حدی کاملا جذب کننده می باشد که در این صورت
$\Gamma_0 \to 1$
لذا
$\xi \to 1$
با محاسبه ی حد رابطه ی 
\ref{fid1m}
در این حالت به دست می آوریم:
\begin{equation}
	\lim_{\xi \to 1} F_{1 \to M} = \frac{2}{3}
\end{equation}

می توان در حالت کلی نشان داد
\cite{cloner}
که برای حالت 
$N \to M$
وفاداری در حالت کاملا جذب کننده برابر است با 
\begin{equation}
	F_{N \to M} = \frac{N+1}{N+2}
\end{equation}
که نشان می دهد این کاملا مستقل از
$\omega/T$
می باشد.


\subsection{بررسی نتایج}


در این قسمت نتیجه گرفتیم که یک سیاه چاله می تواند به عنوان یک کلونر عمل کند که با محاسبه ی وفاداری آن نشان دادیم که وفاداری
$1 \to M$
سیاه چاله به صورت
\begin{equation}
F_{1 \to M} = \frac{3+\xi + 2\xi M}{3(2+\xi M)}
\end{equation}
می باشد. این رابطه نشان می دهد که سیاه چاله در حالت حدی کاملا بازتاب کننده
$\xi \to \infty$
به یک کلونر بهینه تبدیل می شود و بیشترین وفاداری ممکن از نظر تئوری را ایجاد می کند. همچنین در حالت حدی کاملا جذب کننده این رابطه به 
$\xi$
یا
$\omega/T$
وابسته نمی باشد. این رابطه در حالتی که 
$\omega/T$
مقدار ثابت 
$4.0$
را داشته باشد، به ازای 
$\Gamma_0$
های مختلف بر حسب 
$M$
در شکل 
\ref{fig441}
رسم شده است. همچنین در شکل
\ref{fig442}
این بار به ازای 
$\Gamma_0 = 0.95$
ثابت و به ازای 
$\omega/T$
های مختلف بر حسب
$M$
رسم شده است. مشاهده می شود که در هر حالت، با افزایش 
$M$
وفاداری کاهش می یابد و در حالتی که 
$M \to \infty$
همه ی این منحنی ها به حالت کاملا جذب کننده
$F_{1 \to M} = 2/3$
میل می کنند.

\begin{figure}
	\centering
	\includegraphics[width=.7\linewidth]{\pngPATH{4}{1}}
	\caption{نمودار وفاداری سیاه جاله بر حسب $M$ به ازای $\Gamma_0$ های مختلف و در $\omega/T=4.0$ ثابت \cite{cloner}}
	\label{fig441}
\end{figure}

\begin{figure}
	\centering
	\includegraphics[width=.7\linewidth]{\pngPATH{4}{2}}
	\caption{نمودار وفاداری سیاه جاله بر حسب $M$ به ازای $\omega/T$ های مختلف و در $\Gamma_0=0.95$ ثابت \cite{cloner}}
	\label{fig442}
\end{figure}
