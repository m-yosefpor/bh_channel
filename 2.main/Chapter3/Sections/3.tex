\section{ظرفیت کوانتمی سیاه چاله}
%%%%%%%%%%%%%%%%%%%%%%%%%%%%%%%%%%%%%%%%%%%%%%%%%%%%%%%%%%%%%%%%%%%%%%%
\subsection{محاسبه ی ظرفیت  کوانتمی تابش هاوکینگ}
در این قسمت ظرفیت کوانتمی کانالی را بررسی می کند که خروجی اش تابش هاوکینگ می باشد. لذا مجددا در اینجا هم مشابه رابطه
\ref{isomw}
با در نظر نگرفتن قسمت مربوط به دیرهنگام ها برای ایزومتری کانال داریم:
\begin{equation} \label{isomrw}
V(r_\omega) =  \prod_{\omega} e^{ r_\omega (a_k^\dagger b_{-k}^\dagger - a_k b_{-k})}
\end{equation}

که در این رابطه،
$r_\omega$
مربوط به مود
$\omega$
بر حسب جاذبه سطح 
$\kappa = 1/2M$
و
$M$
جرم سیاه چاله، برابر است با 
\cite{qit}
:
\begin{equation}
	r_\omega = \tanh ^{-1} (e^{-\pi \omega / \kappa})
\end{equation}

در نتیجه با استفاده از رابطه ی 
\ref{vout} 
داریم
\cite{qit}:

\begin{equation} \label{qvout}
	V^{A \to BE}(r_\omega) |vac\rangle = \prod_\omega \frac{1}{\cosh r_\omega} \sum_{n=0}^{\infty} \tanh ^n r_\omega |n\rangle_B |n\rangle_E = \prod_\omega \sigma_{\omega, BE}
\end{equation}

به سادگی می توان نشان داد که 
\begin{equation}
	tr_E (\sigma_{\omega, BE}) = tr_B (\sigma_{\omega, BE})
\end{equation}
لذا کانال حاصل یک کانال کوانتمی متقارن
\LTRfootnote{Symmetric}
می باشد و لذا ظرفیت آن صفر است
\cite{qit}.

پس ظرفیت انتقال اطلاعات کوانتمی با تابش هاوکینگ صفر می باشد و لذا نمی توان از این تابش اطلاعات مربوط به جسم اولیه را بازیابی کرد.
%%%%%%%%%%%%%%%%%%%%%%%%%%%%%%%%%%%%%%%%%%%%%%%%%%%%%%%%%%%%%%%%%%%%%%%
\subsection{محاسبه ی ظرفیت  کوانتمی تابش برانگیخته در مود های زودهنگام}

در این قسمت به محاسبه ی ظرفیت کوانتمی تابش برانگیخته می پردازیم. ابتدا با تغییر رویکرد از دیدگاه تصویر هایزنبرگ به تصویر شرودینگر  تبدیل 
\lr{Bogoliubov}
را می توان به صورت زیر نوشت
\cite{qit}
:

\begin{equation}
	|n\rangle \to \prod_\Omega \frac{1}{\cosh^{1+n} r_\Omega} \sum_{m=0}^{\infty} \binom{n+m}{n} ^{1/2} \tanh^m r_\Omega |n+m\rangle_{\Omega,L} |m\rangle_{\Omega,R}
\end{equation}

که در آن 
\begin{equation}
r_\Omega = \tanh ^{-1} (e^{-\pi \Omega})
\end{equation}
می باشد و 
$\Omega$
فرکانس مینکوفسکی اسکیل شده 
$\Omega = \omega / a$
است.


اگر حالت 
$n$
ذره  
$|n\rangle$
را در معادله  
\ref{qvout}
جایگزین کنیم، ایزومتری کانال را به دست می آوریم:

\begin{equation}
	V_\Omega^{A \to BE} |n\rangle_A = \frac{1}{\cosh^{1+n} r_\Omega} \sum_{m=0}^{\infty} \binom{n+m}{n} ^{1/2} \tanh^m r_\Omega |n+m\rangle_{B} |m\rangle_{E}
\end{equation}

لذا خروجی کانال را با اثر جزئی گیری به صورت زیر به دست می آوریم:
\begin{equation}
	\cn = tr_E (V_\Omega ^{A \to BE})
\end{equation}
می توان نشان داد 
\cite{qit}
که این خروجی کانال را می توان به صورت زیر نوشت:

\begin{equation} \label{unrup}
	\cn = \bigoplus_{l=1}^{\infty} p_l N_l
\end{equation}
که 
$p_l$
یک توزیع احتمال است:
\begin{align}
	p_l &= \frac{1}{2} (1-z)^3 l (l+1) z^{l-1} \\
	z &= \tanh^2 r_\Omega = \exp (-2\pi \omega /a )  
\end{align}

به چنین کانالی، کانال جمع مستقیم
\LTRfootnote{Direct Sum Channel}
یا کانال مخلوط احتمالی
\LTRfootnote{Probabilistic Mixture of Channels}
می گویند.

در رابطه ی 
\ref{unrup}
، 
$\cn_l$
همان
$Cl_{1,l}$
 کلون کننده
$1$
به 
$n$
می باشد. یعنی 
\begin{equation} \label{unru}
\cn = \bigoplus_{l=1}^{\infty} p_l N_l
\end{equation}


در واقع کانال های
$Cl_{n,l}$
کلون کننده 
$n$
به
$l$
است و در واقع از 
$n$
حالت نامعلوم ورودی
$l$
کلون و 
$l-n$
پادکلون
\LTRfootnote{Anti-Clone}
درست می کند. می توان نشان داد 
\cite{qit}
که این کانال ها دسته ای از کانال های نازل شونده
\LTRfootnote{degradable}
می باشند.


کانال 
$\cn$
یک کانال نازل شونده
است اگر و فقط اگر یک کانال دیگری مانند
$D$
وجود داشته باشد، به طوری که 
$\widehat{\cn} = D \circ \cn$
که 
$\widehat{\cn}$
مکمل کانال 
$\cn$
می باشد. می توان نتیجه گرفت که برای این کانال ها رابطه ی زیر برقرار است
\cite{chuang,wilde}
:
\begin{equation} \label{degr}
		Q^{(1)} (\cn \otimes \cn) \leq 2 Q^{(1)} (\cn)
\end{equation}

از طرفی با توجه به تعریف ظرفیت کوانتمی بدیهی است که 
\begin{equation}
	Q^{(1)} (\cn \otimes \cn) \geq 2 Q^{(1)} (\cn)
\end{equation}
 همچنین با در نظر گرفتن رابطه ی
\ref{degr}
داریم:
\begin{equation}
	Q^{(1)}(\cn \otimes \cn) = 2 Q^{(1)} (\cn)
\end{equation}
در نتیجه با استقرا به دست می آوریم:
\begin{equation}
Q^{(1)}(\cn ^{\otimes n}) = n Q^{(1)} (\cn)
\end{equation}

با جایگذاری این در رابطه 
\ref{qdef}
نتیجه می گیریم که ظرفیت کانال های  نازل شونده
تک حرفی است و
\begin{equation}
	Q(\cn) = Q^{(1)} (\cn)
\end{equation}

از طرفی، با توجه به این که کلونر ها از کانال های نازل شونده
است، ظرفیت آن قابل محاسبه است. نشان داده می شود که ظرفیت کانال
$Cl_{1,l}$
به صورت زیر است:
\begin{equation} \label{qcl}
	Q(Cl_{1,l}) = \log_2 \frac{l+1}{l}
\end{equation}

به کانال
\ref{unru}
که به صورت جمع مستقیمی از کانال های کلون کننده می باشد کانال
\lr{Unruh}
می گویند.

با جایگذاری  رابطه ی
\ref{qcl}
در رابطه ی 
\ref{unru}
ظرفیت این کانال 
\lr{Unruh}
به صورت زیر قابل محاسبه می باشد:
\begin{equation} \label{unrq}
	Q(\cn) = \frac{1}{2} (1-z)^3 \sum_{l=1}^{\infty} l (l+1) z^{l-1} \log_2 \frac{l+1}{l}
\end{equation}
%%%%%%%%%%%%%%%%%%%%%%%%%%%%%%%%%%%%%%%%%%%%%%%%%%%%%%%%%%%%%%%%%%%%%%%
\subsection{محاسبه ی ظرفیت کوانتمی تابش برانگیخته در مود های دیرهنگام} %faghat halat hadi toonestim
در اینجا رابطه ی 
\ref{param}
را با در نظر گرفتن
$\gamma$
لحاظ می کنیم. این معادله را فقط در دو حالت حدی بررسی می کنیم، یکی در حالت بازتاب کننده کامل 
\LTRfootnote{Perfectly Reflecting}
($\alpha = 0$)
و دیگری جذب کننده کامل
\LTRfootnote{Perfectly Absorbing} 
($\alpha = 1$).

\subsubsection{حالت بازتاب کننده کامل}
در این حالت معادله 
\ref{param}
به صورت زیر در می آید:
\begin{equation}
A_k = \gamma c_k - \beta b_{-k}^\dagger
\end{equation}

مشاهده می شود که این رابطه کاملا مشابه حالت زود هنگام می باشد با این تفاوت که در اینجا نقش
$a$
را
$c$
و نقش
$\alpha$
را
$\gamma$
بازی می کند. همچنین رابطه ی 
\ref{paran}
به رابطه ی زیر کاهش می یابد:
\begin{equation}
\gamma_k^2 - \beta_k ^2  = 1
\end{equation}

لذا دقیقا مشابه قبل نتیجه می گیریم که
\begin{equation}
\cn = \bigoplus_{l=1}^{\infty} p_l Cl_{1,l}
\end{equation} 

پس ظرفیت این کانال 
\lr{Unruh}
نیز مطابق رابطه 
\ref{unrq}
می باشد.

یکی از ویژگی های جالب کانال
\lr{Unruh}
این است که مکمل آن یک کانال درهم تنیدگی شکننده
\LTRfootnote{Entanglement-Breaking Channel}
 است.
 
  کانال کوانتمی
$M$
درهم تنیدگی شکننده است، اگر و فقط اگر
$\varrho ^{AB} = (id^A \otimes M^B) (\phi^{AB})$
برای همه ی حالت های دوگانه
\LTRfootnote{Bipartite}
$\phi^{AB}$
جدا شونده باشد. می توان نشان داد
\cite{qit}
که اگر کانالی در هم تنیدگی شکننده باشد، ظرفیت آن صفر می باشد. بنابرین هیچ اطلاعاتی را نمی توان به صورت قابل اطمینان از طریق کانال درهم تنیدگی شکننده ارسال کرد. عکس این قضیه بر قرار نمی باشد؛ یعنی لزوما تمام کانال های با ظرفیت صفر در هم تنیدگی شکننده نیستند و کانال هایی وجود دارند که با اینکه دارای ظرفیت صفر هستند، اما در هم تنیدگی شکننده نمی باشند.

لذا اگر بخواهیم در بالا، اطلاعات فرستاده شده به داخل سیاه چاله را بررسی کنیم، این بار باید اثر جزئی را روی محیط
$B$
را محاسبه کنیم. یعنی 
\begin{equation}
	\cn ' = tr_B (V^{A \to BE})
\end{equation}

که دقیقا همان مکمل کانال 
$\cn$
می باشد. لذا
\begin{equation}
	\cn ' = \widehat{\cn} = \bigoplus_{l=1}^{\infty} p_l \widehat{Cl}_{1,l}
\end{equation}

که در آن 
$\widehat{Cl}_{1,l}$
نشان دهنده مکمل یک کانال کلون کننده می باشد. این کانال 
$l$
پادکلون از ورودی نامعلوم درست می کند.

صفر بودن ظرفیت این کانال کوانتمی به این معنی است که از این پاد کلون های ارسال شده به داخل کانال نمی توان اطلاعات جسم برخورد کرده را بازیابی کرد.

\subsubsection{حالت جذب کننده کامل}
با توجه به اینکه در این حالت
$\alpha=1$
می باشد، لذا :

\begin{equation}
	\gamma^2 = \beta^2 = g_\omega^2
\end{equation}

پس می توان رابطه ی 
\ref{param}
را به صورت زیر نوشت:
\begin{equation}
A = a - g_\omega ( b^\dagger + c)
\end{equation}

در این حالت نیز به محاسبه ایزومتری کانال می پردازیم. به روشی ساده ولی طولانی می توان نشان داد 
\cite{qit}
که اگر اطلاعات به صورت دوخطی
\LTRfootnote{Dualrail}
کد شود، می توان ایزومتری کانال را به صورت زیر نوشت:

\begin{equation}
	V_\omega |000\rangle_{abc} = \frac{2}{2+g_\omega^2} \sum_{n=0}^{\infty} \sum_{k=0}^{n} A^{n-k} B^{k} \sqrt{\binom{n}{k}} |n-k\rangle_a |n\rangle_b |k\rangle_c
\end{equation}

و 
\begin{align}
	V_\omega |000\rangle_{abc} &= (\frac{2}{2+g_\omega^2})^2 \sum_{n=0}^{\infty} \sum_{k=0}^{n+1} A^{n-k} B^{k} \sqrt{\binom{n}{k}} \\
	& \times  (\sqrt{k+1} |n-k\rangle_a |n\rangle_b |k+1\rangle_c  + g_\omega \sqrt{n-k+1} |n-k+1\rangle_a |n\rangle_b |k\rangle_c )
\end{align}

که در آن
\begin{equation}
	A= \frac{2g_\omega}{2+g_\omega^2}
\end{equation}
و
\begin{equation}
	B= - \frac{g_\omega^2}{2+g_\omega^2}
\end{equation}

با اثر جزئی گرفتن از ایزومتری به صورت زیر می توان خروجی کانال را محاسبه کرد:

\begin{equation}
\mathcal{M} = tr_E (V_\Omega ^{A \to BE})
\end{equation}
می توان نشان داد 
\cite{qit}
که این خروجی کانال را می شود به صورت زیر نوشت:

\begin{equation} 
\mathcal{M} = \bigoplus_{l=1}^{\infty} p_l D_l
\end{equation}
که 
$p_l$
یک توزیع احتمال است. در این کانال، 
$D_l$
ها کلون کننده نیستند و لذا کانال حاصل کانال
\lr{Unruh}
نمی باشد و با آن تفاوت دارد. می توان نشان داد
\cite{qit}
که 
$D_l$
کانال 
\lr{depolarizing}
می باشد.

کانال های 
\lr{depolarizing}
کانال هایی هستند که می توان آن ها را به صورت زیر نوشت
\cite{chuang}
:

\begin{equation}
	D_1(\varrho) = (1-q) \varrho + \frac{q}{2} id_2
\end{equation}

که برای 
$ 0 \leq q \leq 4/3$
تعریف شده اند. خاصیت کانال های
\lr{depolarizing}
به شدت به پارامتر 
$q$
وابسته می باشد. در حالت کانال سیاه چاله کاملا جذب کننده 
$q$
را برابر
$2/3$
پیدا می کنیم. در این حالت اگر 
$\phi^{AB}$
یک حالت مخلوط بیشینه باشد، آن گاه ماتریس چگالی خروجی 
$\varrho^{AB} = (id^A \otimes D_1 ^B ) (\phi^{AB})$
مثبت معین است. این شرط لازم و کافی است تا بتوان نتیجه گرفت
\cite{chuang}
که خروجی
$\varrho$
جدا شونده می باشد و لذا این کانال در هم تنیدگی شکننده است. پس ظرفیت آن صفر است. این یک نتیجه مهم برای کانال سیاه چاله 
$\mathcal{M}$
به دنبال دارد که خود
$\mathcal{M}$
هم یک کانال در هم تنیدگی  شکننده می باشد
\cite{qit}
و لذا ظرفیت آن صفر می باشد. پس از تابش های بر انگیخته در مود های دیر هنگام نمی توان اطلاعات مربوط به جسمی که به سمت سیاه چاله فرستاده شده است را بازیابی کرد.



%%%%%%%%%%%%%%%%%%%%%%%%%%%%%%%%%%%%%%%%%%%%%%%%%%%%%%%%%%%%%%%%%%%%%%%%
\subsection{بررسی نتایج}
در ابتدای این بخش ظرفیت کوانتمی کانال سیاه چاله را توسط تابش هاوکینگ مورد بررسی قرار دادیم و نشان دادیم که این کانال یک کانال کوانتمی متقارن است و لذا ظرفیت آن صفر است. برای تابش برانگیخته، اثبات کردیم که به صورت یک کانال
 \lr{Unruh}
می باشد و ظرفیت آن برابر 
\begin{equation}
Q(\cn) = \frac{1}{2} (1-z)^3 \sum_{l=1}^{\infty} l (l+1) z^{l-1} \log_2 \frac{l+1}{l}
\end{equation}
است. که در آن
\begin{align}
	p_l &= \frac{1}{2} (1-z)^3 l (l+1) z^{l-1} \\
	z &= \tanh^2 r_\Omega = \exp (-2\pi \omega /a )  
\end{align}

\begin{figure}
	\centering
	\includegraphics[width=0.7\linewidth]{\pngPATH{3}{2}}
	\caption{نمودار ظرفیت کوانتمی تابش برانگیخته در مود های دیر هنگام  (ظرفیت کانال \lr{Unruh}) بر حسب پارامتر $z$ \cite{qit}}
	\label{fig332}
\end{figure}

نمودار ظرفیت کانال بر حسب پارامتر 
$z$
در شکل
\ref{fig332}
رسم شده است. همان طور که مشاهده می شود، ظرفیت این کانال با افزایش
$z$
کاهش می یابد و در حد 
$z \to 1$
به صفر میل میکند. اما این حالت حدی از نظر فیزیکی بی معنی می باشد
\cite{qit}
و لذا در همه جا ظرفیت کانال مثبت است. 

سپس مود های دیر هنگام را در دوحالت حدی بررسی کردیم، ابتدا در حالت کاملا بازتاب کننده دیدیم  که کانال ایجاد شده توسط تابش های بر انگیخته مجددا به کانال 
\lr{Unruh}
منجر می شود و لذا همان طور که گفته شد دارای ظرفیت غیر صفر است.

اگر سیاه چاله را کاملا جذب کننده در نظر بگیریم کانال کوانتمی حاصل، یک کانال 
\lr{depolarizing}
می باشد که ظرفیت آن صفر است.

مشاهده می شود که ظرفیت در نمودار
\ref{fig332}
نسبت به حالت کلاسیکی آن در نمودار
\ref{fig333}
کمتر است.

%%%%%%%%%%%%%%%%%%%%%%%%%%%%%%%%%%%%%%%%%%%%%%%%%%%%%%%%%%%%%%%