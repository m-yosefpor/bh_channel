\section{سیاه چاله ها}

سیاه چاله ها، اجسامی با چگالی بسیار زیاد هستند که میدان گرانشی چنان قوی ای دارند که سرعت فرار هر ذره داخل سطح بسته ای دور آن ها (که به آن افق رویداد 
\LTRfootnote{Event Horizon}
می گویند) از سرعت نور بیشتر است.  این موضوع متضمن این است که هیچ جسم فیزیکی، از جمله خود نور، وقتی وارد آن شد نمی تواند از آن خارج شود.  به بیان دیگر، مخروط نوری آینده 
\LTRfootnote{Future Light Cone}
هر نقطه درون افق رویداد، به تمامی داخل آن قرار می گیرد (شکل 
\ref{figcone}
). هر سیگنالی که توسط نقطه ای درون سیاه چاله فرستاده می شود، به اعماق سیاه چاله به سمت نقطه ی تکینی 
\LTRfootnote{Singularity}
آن حرکت می کند.


\begin{figure}
	\centering
	\includegraphics[width=.37\linewidth]{\pngPATH{1}{2}}
	\caption{مخروط نوری آینده، هر نقطه درون سیاه چاله تماما داخل آن قرار می گیرد. \cite{slides}}
	\label{figcone}
\end{figure}


در اینجا از توصیف تخصصی سیاه چاله ها که به نسبیت عام نیاز دارد صرف نظر شده است، اگرچه برای بررسی مدل ریاضی آن، از مدل شبه کلاسیکی 
\lr{Bogoliubov}
آن که توسط رافائل سورکین
\LTRfootnote{Rafael D. Sorkin}
\cite{sorkin}
ارایه شده است استفاده می کنیم.

