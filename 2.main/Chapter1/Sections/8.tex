\section{ساختار پایان نامه و علامت گذاری}
%
%
در این پایان نامه، ابتدا به مبانی ریاضی مورد نیاز برای بررسی این مسئله در فصل 2 پرداخته می شود. این فصل شامل مفاهیمی از تئوری اطلاعات کلاسیک، مکانیک کوانتمی، تئوری اطلاعات کوانتمی و بررسی برخی از کانال های خاص در مکانیک کوانتمی می باشد. 
سپس به مدل ریاضی سیاه چاله به عنوان یک کانال کوانتمی پرداخته شده و ظرفیت کلاسیکی (ظرفیت  
\lr{Holevo})
و ظرفیت کوانتمی آن کانال محاسبه شده و در ادامه فصل به وفاداری
\LTRfootnote{Fidelity}
آن  کانال توجه می کنیم. در هر بخش، نتایج به دست آمده را در راستای پارادوکس اطلاعات سیاه چاله ها تجزیه و تحلیل می کنیم و اثر تغییر هر پارامتر بر ظرفیت سیاه چاله را با کمک نرم افزار متلب
\LTRfootnote{MATLAB}
بررسی می کنیم.
 در آخر نیز، به جمع بندی مطلب می پردازیم و پیشنهادات موجود برای ادامه کار تبیین می شود.
 
 در این پایان نامه، متغیر های تصادفی با حروف بزرگ 
  نمایش داده شده اند (به طور مثال 
 $X$).
  همچنین عملگر ها و ماتریس ها به جز ماتریس های چگالی با حروف بزرگ نشان داده شده اند  (به طور مثال
  $U$
  نمایانگر یک عملگر و 
  $\rho$
  نمایانگر یک ماتریس چگالی می باشد
   ) و 
  برای بردار ها نمایش برا-کتی دیراک (به عنوان مثال 
  $\pk$
  ) در نظر گرفته شده است. همچنین، کانال ها را با حروف بزرگ انحنا دار (به عنوان مثال 
  $\cn$)
  نشان می دهیم.