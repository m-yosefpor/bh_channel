\section{توجیه پارادوکس با رویکرد تئوری اطلاعات}
روش های متعددی برای رفع این پارادوکس بیان شده است که در اکثر آن ها مشکل را به درک ناکامل ما از گرانش کوانتمی ارتباط داده اند. اما در رویکرد تئوری اطلاعاتی، منشا پارادوکس را در نحوه ی صحیح بیان مساله، با یک فرمالیسم مناسب که با تئوری اطلاعت بیان می شود، مرتبط می دانند
\cite{cit, qit}.

با توجه به اینکه یک تحول زمانی، در واقع حالت یک سیستم را از حالت اولیه به حالت ثانویه تغییر می دهد لذا در این روش می توان سیاه چاله را به عنوان یک نگاشت یا به بیان دیگر به عنوان یک کانال مخابراتی در نظر گرفت که حالت اولیه ورودی کانال، و حالت ثانویه خروجی این کانال می باشد. پس اگر پس از مدل کردن تحول سیاه چاله، ظرفیت آن را محاسبه کنیم، می توانیم به این نتیجه برسیم که آیا اطلاعات در این سیستم از بین می رود یا خیر.

ورودی کانال را می توان حالت اولیه ای که تشکیل سیاه چاله می دهد، یا جسمی که پس از تشکیل سیاه چاله به درون آن جذب می شود در نظر گرفت. در این صورت می توان سرنوشت اطلاعات موجود در سیستم مورد نظر را مطالعه کرد. خروجی کانال را می توان تابش هاوکینگ یا تابش بر انگیخته و حتی حالت پایانی سیاه چاله پس از تبخیر کامل در نظر گرفت. اگر ظرفیت کانال با ورودی و خروجی مورد نظر صفر شد، لذا اطلاعات سیستم اولیه در خروجی مورد نظر (مثلا تابش هاوکینگ یا ...) وجود ندارد و نمی توان از آن خروجی اطلاعات سیستم اولیه را بازیابی کرد.

در ادامه با بررسی کانال های کلون کننده، پارادوکس کلون شدن اطلاعات بررسی می شود که آیا ممکن است اطلاعات در یک زمان، هم درون سیاه چاله و هم خارج آن وجود داشته باشد؟ به عبارتی آیا می توان از تابش ها و مود های منتشره در خارج سیاه چاله و در داخل آن، اطلاعات سیستم اولیه را بازیابی کرد؟ در این پایان نامه، با بررسی ظرفیت  کلاسیکی و ظرفیت کوانتمی کانال مدل شده ی سیاه چاله با در نظر گرفتن ورودی ها و خروجی های متفاوت، و همچنین محاسبه ی وفاداری سیاه چاله در حالات حدی به  توجیه این پارادوکس پرداخته می شود. 