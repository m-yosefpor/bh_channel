\section{سایر راه حل هاي ارایه شده در جهت توجیه پارادوکس}
برخی از دانشمندان علت  وجود پارادوکس را ناشی از درک ناکامل ما از گرانش کوانتمی دانسته اند. به عنوان مثال، لئونارد ساسکیند
\LTRfootnote{Leonard Susskind}
بیان کرده است که ممکن است  نتوان فضای داخل و خارج افق رویداد را به عنوان دو زیر سیستم از یک سیستم نگاه کرد. به بیان دقیق تر، توصیف فضای هیلبرت سیستم کل را نمی توان به صورت ضرب تانسوری فضای هیلبرت داخل سیاه چاله و فضای هیلبرت خارج آن نوشت. بلکه فضای درون و خارج افق رویداد، هر دو توصیف کاملی از یک سیستم یکسان هستند که هر توصیف بستگی به این دارد که ناظر داخل سیاه چاله است یا خارج آن. این دیدگاه به مکملی سیاه چاله
\LTRfootnote{Black Hole Complementarity}
 معروف است
\cite{SusComplementarity}. %susskind 1993



روش های دیگری نیز وجود دارند، مانند بررسی سیاه چاله ها از نظر زمان نگهدارندگی اطلاعات
\LTRfootnote{Information Retention Time}
\cite{mirror}
، بررسی به عنوان اسکرمبلر های سریع
\LTRfootnote{Fast Scrambler}
\cite{SusScramb} %seiko and suskind 2008
یا بررسی از دیدگاه نظریه ی ریسمان یا نظریه ی 
$M$
\LTRfootnote{String Theory, M-Theory}
که بنیادی ترین ذرات تشکیل دهنده را به صورت اجسام گسترده دارای ابعاد متفاوت (ریسمان) در نظر می گیرد.

در دیگاه دیگر، که به 
\lr{AdS-CFT}
معروف است، به توصیف هولوگرافیک نظریه ی ریسمان می پردازد. فضای
\lr{Anti-de Sitter (AdS)}
یک فضازمان با انحنای منفی ثابت می باشد. اگرچه مقاطع مکانی نامحدود اند، این فضازمان دارای مرز می باشد. در این روش نظریه ی ریسمان روی 
$d+1$
بعد، به صورت دقیق با یک نظریه ی میدان مشترک
\LTRfootnote{Conformal Field Theory (CFT)}
$d$
بعدی که روی مرز فضازمان تعریف شده است، توصیف می شود.  می توان نشان داد که تحولات در این نظریه ی میدان دوگان، یکانی آشکار هستند و لذا اطلاعات در آن نابود نمی شود. پس می توان انتظار داشت که در مسئله ی دوگان آن که همان سیاه چاله است،  نحوه کار گرانش کوانتمی بر این مبنا باشد که سیاه چاله اطلاعاتی را نابود نکند
\cite{AdS}. %maldacena 1997