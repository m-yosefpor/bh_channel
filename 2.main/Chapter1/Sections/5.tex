\section{پارادوکس اطلاعات سیاه چاله}
طبق قوانین مکانیک کوانتمی، تحول هر سیستم ایزوله، توسط یک عملگر یکانی در فضای هیلبرت مشخص می شود
\cite{shankar}
این عملگر تحول همان هامیلتونی سیستم است.  از ویژگی های عملگر های یکانی این است که همیشه مزدوجی دارند که ترکیب آن ها عملگر همانی (یکه) می شود. به بیان دیگر هرگاه یک سیستمی طبق عملگر یکانی  از حالت یک به حالت دو تحول پیدا میکند، همیشه عملگر یکانی دیگری وجود دارد که می تواند حالت دو را به حالت یک برگرداند. به عبارتی تحول یک سیستم همیشه از نظر میکروسکوپی بازگشت پذیر خواهد بود
\cite{intro}
. 

حال فرض کنید سیستمی که یک سری اطلاعات اولیه داخلش است شروع به فروپاشی گرانشی می کند تا به سیاه چاله تبدیل شود.
پس از اینکه جسم به سیاه چاله تبدیل شد و شروع به تابش هاوکینگ کرد، از آنجایی که تابش هاوکینگ یک تابش بدون ویژگی
\LTRfootnote{Featureless}
است و هیچ وابستگی به اطلاعات کد شده در حالت اولیه ی سیستم ندارد،  حاوی هیچ اطلاعاتی نیست. پس از تبخیر کامل و نابودی سیاه چاله اطلاعات اولیه موجود در سیستم چه می شود؟!  مساله اساسی اینجاست که اگر چه اجسام داغ دیگر هم تابش گرمایی دارند و از نظر ترمودینامیکی غیر برگشت پذیر هستند، اما از نظر میکروسکوپی همیشه قابل برگشت هستند. اما سیاه چاله با تمام اجسام داغ دیگر تفاوت اساسی دارد، به این خاطر که دارای افق رویداد است. این بدین معنی است که تحول ذکر شده از نظر میکروسکوپی غیر قابل برگشت است. لذا اطلاعات نه فقط از نظر عملی، بلکه از نظر اصولی گم شده است!
\begin{figure}
	\centering
	\includegraphics[height=.5\linewidth]{\pngPATH{1}{3}}
	\caption{اطلاعات کد شده در حالت اولیه جسم، بعد از تبخیر کامل سیاه چاله چه می شود؟ \cite{slides}}
	\label{fig3}
\end{figure}

اما تابش هاوکینگ تنها تابش یک سیاه چاله نیست. تابش دیگری به نام تابش بر انگیخته
\LTRfootnote{Stimulated Emission} 
 از سیاه چاله وجود دارد که باید این تابش نیز از نظر محتوی اطلاعاتی بررسی شود. اما پارادوکس فقط به اینجا ختم نمی شود. حتی اگر تابش برانگیخته حاوی اطلاعات سیستم اولیه باشد، می توان نشان داد زمان هایی وجود دارد که اطلاعات سیستم، هم در جسم در حال سقوط به تکینگی سیاه چاله و هم در تابش های سیاه چاله وجود دارد.  برای فهم بهتر، شکل
 \ref{fig4}
 را در نظر بگیرید. با توجه به اینکه میدان گرانشی طبق نسبت عام باعث خمیدگی فضا زمان می شوند، لذا میدان گرانشی سیاه چاله باعث می شود تا خطی که در شکل 
 \ref{fig4}
  به عنوان
 \lr{nice-slice}
 نشان داده شده است،
  بیانگر یک زمان مشخص در مکان های متفاوت باشد
  \cite{slides}.
  طبق شکل می توان مشاهده کرد که این خط، جسم در حال فروپاشی پشت افق رویداد  (داخل سیاه چاله)و  بخش اعظم تابش بیرون سیاه چاله را در بر می گیرد. می توان این خط را طوری انتخاب کرد که جسم در حال فروپاشی و هر مقدار دلخواه از تابش سیاه چاله را در بر بگیرد. برای مشاهده دقیق تر این موضوع، می توان به نمودار پنروز
 \LTRfootnote{Penrose Diagram}
 در شکل
 \ref{fig5}
 رجوع کرد. در این نمودار خط
 \lr{nice-slice}
 کاملا مشخص است. واضح است که این خط را میتوان به گونه ای که ذکر شد انتخاب کرد. مشاهده می شود که این خط  در نواحی با انحنای کمتر قرار دارد، لذا توقع می رود که فیریک نیمه کلاسیک برای این مسئله مورد قبول باشد. 
 لذا در یک زمان، اطلاعات کوانتمی یکسان در دو مکان متفاوت وجود دارد! این در صورتی است که در تئوری اطلاعات کوانتمی طبق قضیه ی
 \lr{no-cloning}
 نشان داده می شود که چنین چیزی از نظر اصولی غیر ممکن است
 \cite{ipm}.
 
 \begin{figure}
 	\centering
 	\includegraphics[width=.4\linewidth]{\pngPATH{1}{4}}
 	\caption{جسم در حال فروپاشی و خط \lr{nice-slice} \cite{slides}}
 	\label{fig4}
 \end{figure}
 
 
 پس طبق مطالب گفته شده، یا اطلاعات نابود شده است، یا کلون
 \LTRfootnote{Clone}
  شده است که در هر دو حالت، دچار پارادوکس شده ایم و باید قوانین مکانیک کوانتمی دوباره بررسی شود.

اشکال متفاوتی از پارادوکس در زمان های مختلف ذکر شده است. پارادوکس مطرح شده دیگر، بر این اساس بود که با توجه به اینکه می توان حالت اولیه ی یک سیستم قبل از فروپاشی گرانشی به  سیاه چاله را در حالت خالص
\LTRfootnote{Pure State}
قرار داد، و اینکه تابش هاوکینگ در حالت مخلوط بیشینه است، یک تحول یکانی باعث تبدیل یک حالت خالص به یک حالت مخلوط بیشینه شده است. در گذشته حتی این موضوع هم به عنوان یک پارادوکس تلقی می شد. اما با پیشرفت تئوری اطلاعات کوانتمی، مشخص شد که تحول یک حالت خالص به یک حالت مخلوط بیشینه، با تمامی قوانین مکانیک کوانتمی مطابقت دارد
\cite{qit}.

\begin{figure}
	\centering
	\includegraphics[width=.4\linewidth]{\pngPATH{1}{5}}
	\caption{نمودار پنروز \cite{slides}}
	\label{fig5}
\end{figure}

