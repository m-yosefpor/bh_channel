\section{تابش هاوکینگ}

نظریه ی میدان کوانتمی 
\LTRfootnote{Quantum Field Theory}
پدیده ای به عنوان نوسانات کوانتمی خلا 
\LTRfootnote{Quantum Vacuum Fluctuations}
را مطرح می کند که این پدیده را می توان اختصارا به صورت خلق و فنای 
\LTRfootnote{Creation and Annihilation}
سریع و مکرر زوج های ذره و پادذره
\LTRfootnote{Antiparticle} 
توصیف کرد. هاوکینگ نشان داد که احتمال تولید زوج نزدیک افق رویداد سیاه چاله غیر صفر است
\cite{hr,qit}.
لذا، زوج تولید شده در نزدیکی افق رویداد، قبل از بازترکیب جدا شده و یکی از آن ها داخل سیاه چاله گیر می افتد در حالی که دیگری از سیاه چاله فرار میکند (شکل 
\ref{fighr}
). این تابش ذرات (که به آن تابش هاوکینگ 
\LTRfootnote{Hawking Radiation}
گفته میشود)
 از سیاه چاله انرژی می گیرد و طبق رابطه ی هم ارزی جرم و انرژی معروف انیشتین
($E=mc^2$)
، جرم سیاه چاله شروع به کاهش می کند (به همین علت گاها به این پدیده تبخیر
\LTRfootnote{Black Hole Evaporation}
 سیاه چاله می گویند). این روند آن چنان ادامه پیدا می کند تا افق رویداد سیاه چاله ناپدید می شود
\cite{intro}
.

\begin{figure}
	\centering
	\includegraphics[width=.5\linewidth]{\pngPATH{1}{1}}
	\caption{پدیده ی خلق و فنای زوج های ذره و پادذره که به تابش هاوکینگ منجر می شود. \cite{intro}}
	\label{fighr}
\end{figure}

طبق قضیه ی
\lr{no-hair}
همه ی جواب های معادله ی انیشتین-ماکسول می تواند  سیاه چاله را به طور کامل با سه پارامتر  کلاسیکی قابل مشاهده از خارج (انرژی یا جرم، بار و تکانه زاویه ای)  توصیف کند
\cite{do}.
 لذا تابش خارجی سیاه چاله فقط به هندسه ی آن و این پارامتر ها بستگی دارد.
پس تابش هاوکینگ، یک تابش گرمایی است. به بیان دیگر، هیچ همبستگی بین داخل سیاه چاله و تابش ایجاد شده وجود ندارد. همچنین، این تابش، یک حالت مخلوط بیشینه
\LTRfootnote{Maximally Mixed State}
 است
 \cite{qit}.